% Document Metadata
\documentclass[10pt,letterpaper]{report}
\usepackage[utf8]{inputenc}
% Use for Arial Font \usepackage{helvet}
%  \renewcommand{\familydefault}{\sfdefault}
% Use for Times New Roman Font \usepackage{mathptmx}

\usepackage[none]{hyphenat}
\usepackage{tikz, textcomp, gensymb, graphicx, mathtools, amssymb, amsthm, hyperref, array}
  \hypersetup{
      colorlinks=true,
      linkcolor=blue,
      filecolor=magenta,
      urlcolor=blue,
      }
      \setlength\extrarowheight{6pt}

  \graphicspath{ {/home/lowebang/Pictures/} }
\usepackage[letterpaper]{geometry}
  \geometry{top=1in, bottom=1in, left=1in, right=1in}
\usepackage{fancyhdr}
  \pagestyle{fancy}
  \lhead{}
  \chead{}
  \rhead{Cabrera \thepage}
  \lfoot{}
  \cfoot{}
  \rfoot{\LaTeX}
  \renewcommand{\headrulewidth}{1pt}
  \renewcommand{\footrulewidth}{1pt}
  \setlength\headsep{0.333in}

% Command to Circle String
\newcommand*\circled[1]{\tikz[baseline=(char.base)]{
            \node[shape=circle,draw,inner sep=2pt] (char) {#1};}}

% Command to Set Oval Around String
\newcommand{\mymk}[1]{%
  \tikz[baseline=(char.base)]\node[anchor=south west, draw,rectangle, rounded corners, inner sep=2pt, minimum size=7mm,
  text height=2mm](char){\ensuremath{#1}} ;}

\title{Calculus BC -- Worksheet 1 on Differential Equations}
\author{Craig Cabrera}
\date{13 February 2022}

\begin{document}
\maketitle
\begin{center}
  \textbf{\underline{Relevant Formulas and Notes:}}
\end{center}
Let us define some common notational rules in the following table:
\begin{center}
  \begin{tabular}{|  c | c | c |}
    \hline
    Formulas & Lagrange Notation & Leibniz Notation \\
    \hline
    Function & $f(x)$ or $y$ & $f$ or $y$ \\
    \hline
    First Derivative & $f'(x)$ or $y'$ & $\frac{df}{dx}$ or $\frac{dy}{dx}$ or $\frac{d}{dx}f(x)$ \\
    \hline
    Second Derivative & $f''(x)$ or $y''$ & $\frac{d^{2}f}{dx^{2}}$ or $\frac{d^{2}y}{dx^{2}}$ or $\frac{d^{2}}{dx^{2}}f(x)$ \\
    \hline
    Other Derivatives & $f^{(n)}(x)$ or $y^{(n)}$ & $\frac{d^{n}f}{dx^{n}}$ or $\frac{d^{n}y}{dx^{n}}$ or $\frac{d^{n}}{dx^{n}}f(x)$ \\
    \hline

  \end{tabular}
\end{center}

\noindent For this worksheet, we will convert all forms of Lagrange Notation to Leibniz Notation to simplify integration technique. 

Direct Proportionality of $y$ to variable $t$: 

$$ y=kt $$ \\

Inverse Proportionality of $y$ to variable $t$:

$$ y=\frac{k}{t} $$ \\

Joint Proportionality of $y$ to variables $x$, $t$:

$$ y=kxt $$ \\
\pagebreak 

Work the following on \textbf{\underline{notebook paper}}.
\begin{enumerate}
  \item{$\frac{dy}{dx}=\frac{x-3}{y}$ and $y(2)=-5$ \\}
  
    $y\frac{dy}{dx}=x-3$ \\
    
    $ydy = (x-3)dx$ \\
    
    $\int{y}\,dy=\int{\left(x-3\right)}\,dx$ \\
    
    $\frac{1}{2}y^{2}=\frac{1}{2}x^{2}-3x+C\rightarrow \frac{1}{2}\left(-5\right)^{2}=\frac{1}{2}\left(2\right)^{2}-3\left(2\right)+C\rightarrow C=\frac{25-4+12}{2}=\frac{33}{2}$ \\
    
    $y^{2}=x^{2}-6x+2C$ \\
    
    $y=\pm\sqrt{x^{2}-6x+2C}=\pm\sqrt{x^{2}-6x+33}$ \\
    
    Because the negative value is the only one that satisfies the initial condition $y(2)=-5$, \\ $y=-\sqrt{x^{2}-6x+33}$. \\
    
  \item{$y'=2x\sqrt{y}$ and $y(2)=25$ \\}
  
    $\frac{1}{\sqrt{y}}\frac{dy}{dx}=2x$ \\
    
    $\frac{dy}{\sqrt{y}}=2xdx$ \\
    
    $\int{\frac{dy}{\sqrt{y}}}=\int{2x}\,dx$ \\
    
    $2\sqrt{y}=x^{2}+C\rightarrow 2\sqrt{25}=2^{2}+C\rightarrow C=10-4=6$ \\
    
    $\sqrt{y}=\frac{1}{2}x^{2}+\frac{1}{2}C$ \\
    
    $y=\left(\frac{1}{2}x^{2}+\frac{1}{2}C\right)=\left(\frac{1}{2}x^{2}+3\right)=\frac{1}{4}x^{4}+3x^{2}+9$ \\
    
  \item{$\frac{dy}{dx}=4y^{2}\sec^{2}{\left(2x\right)}$ and $y\left(\frac{\pi}{8}\right)=1$ \\}
  
    $\frac{1}{y^{2}}\frac{dy}{dx}=4\sec^{2}{\left(2x\right)}$ \\
    
    $\frac{dy}{y^{2}}=4\sec^{2}{\left(2x\right)}dx$ \\
    
    $\int{\frac{dy}{y^{2}}}=4\int{\sec^{2}{\left(2x\right)}}\,dx$ \\
    
    $-\frac{1}{y}=2\tan{\left(2x\right)}+C\rightarrow -1=2\tan{\left(\frac{\pi}{4}\right)}+C\rightarrow C=-2-1=-3$ \\
    
    $\frac{1}{y}=-2\tan{\left(2x\right)}-C$ \\
    
    $y=-\frac{1}{2\tan{\left(2x\right)}+C}=-\frac{1}{2\tan{\left(2x\right)}-3}$ \\
    
    \pagebreak
    
  \item{$xy\frac{dy}{dx}=\ln{x}$ and $y(1)=2$ \\} 
  
    $y\frac{dy}{dx}=\frac{\ln{x}}{x}$ \\
    
    $ydy = \frac{\ln{x}}{x}dx$ \\
    
    $\int{y}\,dy=\int{\frac{\ln{x}}{x}}\,dx$ \\
    
    $\frac{1}{2}y^{2}=\frac{1}{2}\ln^{2}{x}+C\rightarrow \frac{1}{2}\left(2\right)^{2}=\frac{1}{2}\ln^{2}{1}+C\rightarrow C=2$ \\
    
    $y^{2}=\ln^{2}{x}+2C$ \\
    
    $y=\pm\sqrt{\ln^{2}{x}+2C}=\pm\sqrt{\ln^{2}{x}+4}$ \\
    
    Because the positive value is the only one that satisfies the initial condition $y(1)=2$, \\ $y=\sqrt{\ln^{2}{x}+4}$. \\
    
  \item{$y'=2x\sec{y}$ and $y(2)=-\frac{\pi}{2}$ \\}
  
    $\frac{1}{\sec{y}}\frac{dy}{dx}=2x$ \\
    
    $\frac{dy}{\sec{y}}=2xdx$ \\
    
    $\cos{y}dy = 2xdx$ \\
    
    $\int{\cos{y}}\,dy=\int{2x}\,dx$ \\
    
    $\sin{y}=x^{2}+C\rightarrow \sin{\left(-\frac{\pi}{2}\right)}=2^{2}+C\rightarrow C=-1-4=-5$ \\
    
    $y=\arcsin{x^{2}+C}=\arcsin{x^{2}-5}$ \\
    
    \pagebreak
    
  \item{$y'-xe^{y}=2e^{y}$ and $y(0)=0$ \\}
  
    $\frac{dy}{dx}=2e^{y}+xe^{y}=\left(2+x\right)e^{y}$ \\
    
    $\frac{1}{e^{y}}\frac{dy}{dx}=2+x$ \\
    
    $\frac{dy}{e^{y}}=\left(2+x\right)dx$ \\
    
    $\int{\frac{dy}{e^{y}}}=\int{\left(2+x\right)\,dx}$ \\
    
    $-\frac{1}{e^{y}}=\frac{1}{2}x^{2}+2x+C$ \\
    
    $-\frac{1}{e^{y}}=\frac{1}{2}x^{2}+2x-C\rightarrow \frac{1}{e^{0}}=\frac{1}{2}*0^{2}-2(0)-C\rightarrow C=-1$ \\
    
    $e^{y}=-\frac{1}{\frac{1}{2}x^{2}+2x+1}$ \\
    
    $y=\ln{\left(-\frac{1}{\frac{1}{2}x^{2}-2x+1}\right)}=-\ln{\left(-\frac{1}{2}x^{2}-2x+1\right)}$ \\
    
  \item{$\frac{dy}{dx}=2xy^{3}\sin{\left(x^{2}\right)}$ and $y(0)=-1$ \\}
  
    $\frac{1}{y^{3}}\frac{dy}{dx}=2x\sin{\left(x^{2}\right)}$ \\
    
    $\int{\frac{dy}{y^{3}}}=2\int{x\sin{\left(x^{2}\right)}}\,dx$ \\
    
    $-\frac{1}{2y^{2}}=-\cos{\left(x^{2}\right)}+C$ \\
    
    $\frac{1}{y^{2}}=2\cos{\left(x^{2}\right)}-2C\rightarrow \frac{1}{\left(-1\right)^{2}}=2\cos{\left(0^{2}\right)}-2C\rightarrow C=\frac{1}{2}$ \\
    
    $y^{2}=\frac{1}{2\cos{\left(x^{2}\right)}-1}$ \\
    
    $y=\pm\sqrt{\frac{1}{2\cos{\left(x^{2}\right)}-1}}$ \\
    
    Because the negative value is the only one that satisfies the initial condition $y(0)=-1$, \\ $y=-\sqrt{\frac{1}{2\cos{\left(x^{2}\right)}-1}}$. \\
    
  \item{$\frac{dy}{dx}=\frac{1}{y^{2}}$ and $y(0)=4$ \\} 
  
    $y^{2}dy=dx$ \\
    
    $\int{y^{2}}\,dy=\int{}\,dx$ \\
    
    $\frac{1}{3}y^{3}=x+C\rightarrow \frac{64}{3}=C$ \\
    
    $y^{3}=3x+64$ \\
    
    $y=\sqrt[3]{3x+64}$ \\
    
    \pagebreak
    
  \item{Find a curve in the $xy$-plane that passes through the point $(0,3)$ and whose tangent line at a point $(x,y)$ has slope $\frac{2x}{y^{2}}$. \\}
  
    $\frac{dy}{dx}=\frac{2x}{y^{2}}$ and $y(0)=3$ \\
    
    $y^{2}dy=2xdx$ \\
    
    $\int{y^{2}}\,dy=\int{2x}\,dx$ \\
    
    $\frac{1}{3}y^{3}=x^{2}+C\rightarrow 9=C$ \\
    
    $y^{3}=3x^{2}+27$ \\
    
    $y=\sqrt[3]{3x^{2}+27}$ \\
    
    \hline
    
  \noindent Write a differential equation to represent the following.  
  \item{The rate of change of a population $y$, with respect to time $t$, is proportional to $t$.\\}
  
    $\frac{dy}{dt}=kt$ \\
  
  \item{The rate of change of a population $P$, with respect to time $t$, is proportional to the cube of the population. \\}
  
    $\frac{dP}{dt}=k\sqrt[3]{P}$ \\
    
  \item{Let $P(t)$ represent the number of wolves in a population at time $t$ years, where $t\geq 0$. The rate of change of the population $P(t)$, with respect to $t$, is directly proportional to $500-P(t)$\\}
  
    $\frac{dP}{dt}=k(500-P(t))$ \\
    
  \item{Water leaks out of a barrel at a rate proportional to the square root of the depth of the water at that time. \\}
  
    $\frac{dh}{dt}=k\sqrt{h(t)}$ \\
    
  \item{Oil leaks out of a tank at a rate inversely proportional to the amount of oil in the tank. \\} 
  
    $\frac{dV}{dt}=\frac{k}{V}$
    
\end{enumerate}
\end{document}
