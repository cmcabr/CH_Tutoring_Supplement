% Document Metadata
\documentclass[10pt,letterpaper]{report}
\usepackage[utf8]{inputenc}
% Use for Arial Font \usepackage{helvet}
%  \renewcommand{\familydefault}{\sfdefault}
% Use for Times New Roman Font \usepackage{mathptmx}

\usepackage[none]{hyphenat}
\usepackage{tikz, textcomp, gensymb, graphicx, mathtools, amssymb, amsthm, hyperref, multicol}
  \hypersetup{
      colorlinks=true,
      linkcolor=blue,
      filecolor=magenta,
      urlcolor=blue,
      }
  \graphicspath{ {/home/lowebang/Pictures/} }
\usepackage[letterpaper]{geometry}
  \geometry{top=1in, bottom=1in, left=1in, right=1in}
\usepackage{fancyhdr}
  \pagestyle{fancy}
  \lhead{}
  \chead{}
  \rhead{Cabrera \thepage}
  \lfoot{}
  \cfoot{}
  \rfoot{\LaTeX}
  \renewcommand{\headrulewidth}{1pt}
  \renewcommand{\footrulewidth}{1pt}
  \setlength\headsep{0.333in}

% Command to Circle String
\newcommand*\circled[1]{\tikz[baseline=(char.base)]{
            \node[shape=circle,draw,inner sep=2pt] (char) {#1};}}

% Command to Set Oval Around String
\newcommand{\mymk}[1]{%
  \tikz[baseline=(char.base)]\node[anchor=south west, draw,rectangle, rounded corners, inner sep=2pt, minimum size=7mm,
  text height=2mm](char){\ensuremath{#1}} ;}

\title{Calculus BC - Worksheet on Euler's Method}
\author{Craig Cabrera}
\date{20 February 2022}

\begin{document}
\maketitle
\begin{center}
  \textbf{\underline{Relevant Formulas and Notes:}}
\end{center}
\noindent Point-Slope Formula:

$$y-y_{1}=m\left(x-x_{1}\right)=f'(x)\Delta x$$ \\

\noindent Euler's Formula:  

$$y_{i+1}=y_{i}+h*f(x_{i}, y_{i}), \text{ where } x_{i+1}=x_{i}+h$$

$$y_{i+1}=y_{i}+\frac{dy}{dx}\Delta x$$ \\
\pagebreak 


Work the following on \textbf{\underline{notebook paper}}, showing all steps.
\begin{enumerate}
  \item{}
    \begin{enumerate}
      \item{Given the differential equation $\frac{dy}{dx}=x+2$ and $y(0)=3$. Find an approximation for $y(1)$ by using Euler's method with two equal steps. Sketch your solution. \\}
        \begin{multicols}{2}
          $y_{i+1}=y_{i}+\frac{dy}{dx}\Delta x$ \\
        
          $y(0.5)=3+(0+2)(0.5)=4$ \\
          
          $y(1)=4+(0.5+2)(0.5)=5.25$ \\
          
          \includegraphics[scale=0.3]{point_slope_q1.png}
        \end{multicols}
        
      \item{Solve the differential equation $\frac{dy}{dx}=x+2$ with the initial condition $y(0)=3$, and use your solution to find $y(1)$. \\}
      
        $\frac{dy}{dx}=x+2$ \\
        
        $dy=(x+2)dx$ \\
        
        $\int{}\,dy=\int{(x+2)}\,dx$ \\
        
        $y=\frac{1}{2}x^{2}+2x+C\rightarrow 3=\frac{1}{2}(0)^{2}+2(0)+C\therefore 3=C \text{ and } y=\frac{1}{2}x^{2}+2x+3$ \\
        
        $y=\frac{1}{2}(1)^{2}+2(1)+3=\frac{1+4+6}{2}=5.5$ \\
        
      \item{The error in using Euler's Method is the difference between the approximate value and the exact value. What was the error in your answer? How could you produce a smaller error using Euler's Method? \\}
      
        $y_{err}=5.25-5.5=-0.25$ \\
        
        There was an error of 0.25 units from using Euler's method. A smaller error may be produced by using smaller increments of $\Delta x$ for smaller steps. \\
        
    \end{enumerate}
    
    \hline
    
  \item{Suppose a continuous function $f$ and its derivative $f'$ have values that are given in the following table. Given that $f(2)=5$, use Euler's Method with two steps of size $\Delta x=0.5$ to approximate the value of $f(3)$. }
  \begin{multicols}{2}
    \begin{center}
      \begin{tabular}{| c | c | c | c |}
        \hline
        $x$ & 2.0 & 2.5 & 3.0 \\
        \hline
        $f'(x)$ & 0.4 & 0.6 & 0.8 \\
        \hline
        $f(x)$ & 5 & 5.2 & 5.5 \\
        \hline
      \end{tabular}
    \end{center} \\
    $y(2.5)=5+0.4(0.5)=5.2$ \\
  
    $y(3)=5.2+(0.6)(0.5)=5.5$ \\
  \end{multicols}

  \pagebreak
  
  \item{The curve passing through $(2,0)$ satisfies the differential equation $\frac{dy}{dx}=4x+y$. Find an approximation to $y(3)$ using Euler's Method with two equal steps. \\}
  
    $y(2.5)=0+(4(2)+0)(0.5)=4$ \\
    
    $y(3)=4+(4(2.5)+4)(0.5)=11$ \\
    
    \hline
    
  \item{The table gives selected values for the derivative of a function $f$ on the interval $-2\leq x\leq 2$. If $f(-2)=3$ and Euler's method with a step-size of 1.5 is used to approximate $f(1)$, what is the resulting approximation? \\}
  \begin{multicols}{2}
      \begin{center}
    \begin{tabular}{| c | c |}
      \hline
      $x$ & $f'(x)$ \\
      \hline
      -2 & -0.8 \\
      \hline
      -1.5 & -0.5 \\
      \hline
      -1 & -0.2 \\
      \hline
      -0.5 & 0.4 \\
      \hline
      0 & 0.9 \\
      \hline
      0.5 & 1.6 \\
      \hline
      1 & 2.2 \\
      \hline
      1.5 & 3 \\
      \hline
      2 & 3.7 \\
      \hline
    \end{tabular}
  \end{center} \\
  
  $y(-0.5)=3+(-0.8)(1.5)=1.8$ \\
  
  $y(1)=1.8+(0.4)(1.5)=2.4$ \\
  \end{multicols}
  
  \hline
  
  \item{Let $y=f(x)$ be the particular solution to the differential equation $\frac{dy}{dx}=x+2y$ with the initial condition $f(0)=1$. Use Euler's method, starting at $x=0$ with two steps of equal size, approximate $f(-0.6)$. \\}
  
    $y(-0.3)=1+(0+2(1))(-0.3)=0.4$ \\
    
    $y(-0.6)=0.4+(-0.3+2(0.4))(-0.3)=0.25$ \\
    
    \pagebreak
    
  \item{(2005 BC 4) \\
  Consider the differential equation $\frac{dy}{dx}=2x-y$.}
  \begin{enumerate}
    \item{On the axes provided, sketch a slope field for the given differential equation at the twelve points indicated, and sketch the solution curve that passes through the point (0,1). \\}
      \begin{multicols}{2}
        Table of Values for $\frac{dy}{dx}$: \\
        \begin{tabular}{| c | c | c | c | c |} 
          \hline
          \textbf{2} & -4 & -2 & 0 & 2 \\
          \hline
          \textbf{1} & -3 & -1 & 1 & 3 \\
          \hline
          \textbf{0} & -2 & 0 & 2 & 4 \\
          \hline
           & \textbf{-1} & \textbf{0} & \textbf{1} & \textbf{2} \\
          \hline
        \end{tabular} \\
              
        \includegraphics[scale=0.3]{point_slope_q6.png}
      \end{multicols}
      
    \item{The solution curve that passes through the point $(0,1)$ has a local minimum at $x=\ln{\left(\frac{3}{2}\right)}$. What is the $y$-coordinate of this local minimum? \\}
    
      $\frac{dy}{dx}=0\therefore 0=2x-y\rightarrow 2x=y\rightarrow y=2\ln{\left(\frac{3}{2}\right)}$ \\
      
    \item{Let $y=f(x)$ be the particular solution to the given differential equation with the initial condition $f(0)=1$. Use Euler's method, starting at $x=0$ with two steps of equal size, to approximate $f(-0.4)$. Show the work that leads to your answer. \\}
    
      $f(-0.2)=1+(2(0)-1)(-0.2)=1.2$ \\
      
      $f(-0.4)=1.2+(2(-0.2)-1.2)(-0.2)=1.52$ \\
      
    \item{Find $\frac{d^{2}y}{dx^{2}}$ in terms of $x$ and $y$. Determine whether the approximation found in part (c) is less than or greater than $f(-0.4)$. Explain your reasoning. \\}
    
      $\frac{d^{2}y}{dx^{2}}=\frac{d}{dx}\left(2x-y\right)=2-\frac{dy}{dx}=2-2x+y$ \\
      
      We notice from our value of the second derivative that values of $x$ are negative and values of $y$ are positive, therefore the function $f(x)$ resides in Quadrant II. In the slope field for the first differential equation (part (a)), we notice that the slope line located at (0, 1) resides below the solution curve. Because the tangent line approximation resides below our function, it can be infered that all approximations within Quadrant II are underestimates of the function. Therefore, $1.52<f(-0.4)$. \\
      
      \pagebreak

  \end{enumerate}
  
  \item{(Modified version of 2009 BC 4) \\
  Consider the differential equation $\frac{dy}{dx}=6x^{2}-x^{2}y$. Let $y=f(x)$ be the particular solution to the given differential equation with the initial condition $f(-1)=2$.}
  \begin{enumerate}
    \item{Use Euler's method with two steps of equal size, starting at $x=-1$, to approximate $f(0)$. Show the work that leads to your answer. \\}
    
      $f(-0.5)=2+(4)(0.5)=4$ \\
      
      $f(0)=4+(6(-0.5)^{2}-(-0.5)^{2}(4))(0.5)=4.25$ \\
      
    \item{Find the particular solution $y=f(x)$ to the differential equation with the initial condition $f(-1)=2$. \\}
    
      $\frac{dy}{dx}=6x^{2}-x^{2}y=x^{2}\left(6-y\right)$ \\
      
      $\frac{dy}{6-y}=x^{2}dx$ \\
      
      $\int{\frac{dy}{6-y}}=\frac{1}{3}\int{x^{3}}\,dx$ \\
      
      $-\ln{\left|6-y\right|}=\frac{1}{3}x^{3}+C\rightarrow -ln{\left|6-2\right|}=\frac{1}{3}(-1)^{3}+C\rightarrow C=\frac{1}{3}-\ln{4}$ \\
      
      $\ln{\left|6-y\right|}=-\frac{1}{3}x^{3}-C$ \\
      
      $6-y=e^{\left(-\frac{1}{3}x^{3}-\frac{1}{3}+\ln{4}\right)}=
      e^{\left(-\frac{1}{3}\left(x^{3}+1\right)+\ln{4}\right)}=
      \left(e^{\ln{4}\right)\left(e^{-\frac{1}{3}}\right)^{x^{3}+1}=
      4\left(\frac{1}{\sqrt[3]{e}}\right)^{x^{3}+1}$ \\
      
      $-y=4\left(\frac{1}{\sqrt[3]{e}}\right)^{x^{3}+1}-6$ \\
      
      $y=6-4\left(\frac{1}{\sqrt[3]{e}}\right)^{x^{3}+1}$ \\
      
  \end{enumerate}
    
\end{enumerate}
\end{document}
