% Document Metadata
\documentclass[10pt,letterpaper]{report}
\usepackage[utf8]{inputenc}
% Use for Arial Font \usepackage{helvet}
%  \renewcommand{\familydefault}{\sfdefault}
% Use for Times New Roman Font \usepackage{mathptmx}

\usepackage[none]{hyphenat}
\usepackage{tikz, textcomp, gensymb, graphicx, mathtools, amssymb, amsthm, hyperref, multicol}
  \hypersetup{
      colorlinks=true,
      linkcolor=blue,
      filecolor=magenta,
      urlcolor=blue,
      }
  \graphicspath{ {/home/lowebang/Pictures/} }
\usepackage[letterpaper]{geometry}
  \geometry{top=1in, bottom=1in, left=1in, right=1in}
\usepackage{fancyhdr}
  \pagestyle{fancy}
  \lhead{}
  \chead{}
  \rhead{Cabrera \thepage}
  \lfoot{}
  \cfoot{}
  \rfoot{\LaTeX}
  \renewcommand{\headrulewidth}{1pt}
  \renewcommand{\footrulewidth}{1pt}
  \setlength\headsep{0.333in}

% Command to Circle String
\newcommand*\circled[1]{\tikz[baseline=(char.base)]{
            \node[shape=circle,draw,inner sep=2pt] (char) {#1};}}

% Command to Set Oval Around String
\newcommand{\mymk}[1]{%
  \tikz[baseline=(char.base)]\node[anchor=south west, draw,rectangle, rounded corners, inner sep=2pt, minimum size=7mm,
  text height=2mm](char){\ensuremath{#1}} ;}

\title{Calculus BC - Worksheet on 7.2 \\
      \large Volume by Cross Sections}
\author{Craig Cabrera}
\date{27 February 2022}

\begin{document}
\maketitle
\begin{center}
  \textbf{\underline{Relevant Formulas and Notes:}}
\end{center}

$$V=\int_{a}^{b}{A(x)}\,dx, \, \, A(x)=(f(x)-g(x))(\text{Given Values of Other Properties})$$ \\

$$V=\int_{a}^{b}{A(y)}\,dy, \, \, A(y)=(f(y)-g(y))(\text{Given Values of Other Properties})$$ \\

Properties of Shapes:  \\

Circles: \\

$$r=\frac{d}{2}, \, \, A=\pi r^{2}$$ \\

Squares (where $h$ is hypotenuse): \\

$$s=\sqrt{\frac{h^{2}}{2}}=\frac{\sqrt{2}h}{2}, \, \, A=s^{2}$$ \\

Equilateral Triangles: \\

$$b=s, \, \, A=\frac{\sqrt{3}s^{2}}{4}=\frac{\sqrt{3}}{4}A_{\text{square}}$$ \\

Semiellipses: \\

$$ab=rh, \, \, A=\frac{1}{2}\pi ab$$ \\

\pagebreak 


\noindent Work the following on \textbf{\underline{notebook paper}}. For each problem, draw a figure, set up an integral, and then evaluate on your \underline{calculator}. Give decimal answers correct to three decimal places. \\

\begin{enumerate}
  \item{Find the volume of the solid whose base is bounded by the graphs of $y=x+1$ and $y=x^{2}-1$, with the indicated cross sections taken perpendicular to the $x$-axis. \\}
  \begin{multicols}{2}
    Intersections of Graphs (Bounds): $a=-1, b=2$ \\
    
    $f(x)-g(x)=x+1-(x^{2}-1)=x+2-x^{2}$ \\
    
    \vfill\null
    
    \columnbreak
    
    \begin{center}
      \includegraphics[scale=0.21]{7.2_q1.png} \\
    \end{center} \\
    \columnbreak
  \end{multicols}
  \begin{enumerate}
    \item{Squares \\}
    
      $V=\int_{a}^{b}{s^{2}}\,dx=\int_{-1}^{2}{\left(x+2-x^{2}\right)^{2}}\,dx=\int_{-1}^{2}{x^{4}-2x^{3}-3x^{2}+4x+4}\,dx=$ \\
      
      $[\frac{1}{5}x^{5}-\frac{1}{2}x^{4}-x^{3}+2x^{2}+4x]_{-1}^{2}=\frac{32+1}{5}-\frac{16-1}{2}-(8+1)+2(4-1)+4(2+1)=\frac{81}{10}=8.1$ \\
      
      Going further, let $I=\int_{a}^{b}{s^{2}}\,dx=\int_{-1}^{2}{\left(x+2-x^{2}\right)^{2}}\,dx=8.1$. \\
      
    \item{Rectangles of height 1 \\}
    
      $V=\int_{a}^{b}{bh}\,dx=\int_{-1}^{2}{(x+2-x^{2})(1)}\,dx=[\frac{1}{2}x^{2}+2x-\frac{1}{3}x^{3}]_{-1}^{2}=\frac{4-1}{2}+2(2+1)-\frac{8+1}{3}=\frac{9}{2}=4.5$ \\
      
      Going further, let $J=\int_{a}^{b}{bh}\,dx=\int_{-1}^{2}{(x+2-x^{2})}\,dx=4.5$. \\
      
    \item{Semiellipses of height 2 (The area of an ellipse is given by the formula $A=\pi ab$, where $a$ and $b$ are the distances from the center to the ellipse to the endpoints of the axes of the ellipse.) \\}
    
      $V=\frac{\pi}{2}\int_{a}^{b}{\frac{dh}{2}}\,dx=\frac{J\pi}{2}\approx7.069$ \\
      
    \item{Equilateral triangles \\}
    
    $V=\int_{a}^{b}{\frac{\sqrt{3}s^{2}}{4}}\,dx=\frac{\sqrt{3}I}{4}\approx3.507$ \\
  \end{enumerate}
  
  \pagebreak
  
  \item{Find the volume of the solid whose base is bounded by the circle $x^{2}+y^{2}=4$ with the indicated cross sections taken perpendicular to the $x$-axis. \\}
  \begin{multicols}{2}
    $x^{2}+y^{2}=4\rightarrow y=\sqrt{4-x^{2}}$ \\
    
    Zeroes (Bounds): $a=-2, b=2$ \\
    
    Because our function results in a semicircle when solving for $y$, let each base be doubled. \\
    
    (Base = $2\sqrt{4-x^{2}}$) \\
    
    \vfill\null
    
    \columnbreak
    
    \begin{center}
      \includegraphics[scale=0.22]{7.2_q2.png} \\
    \end{center} \\
    \columnbreak
  \end{multicols}
  \begin{enumerate}
    \item{Squares \\}
      
      $V=\int_{a}^{b}{(2s)^{2}}\,dx=4\int_{-2}^{2}{4-x^{2}}\,dx=4[4x-\frac{1}{3}x^{3}]_{-2}^{2}=4(8+8)-4\left(\frac{8+8}{3}\right)=\frac{128}{3}\approx 42.667$ \\
      
      Going further, let $I=\int_{a}^{b}{(2s)^{2}}\,dx=\int_{-2}^{2}{(2\sqrt{4-x^{2}})^{2}}\,dx=42.667$ \\
      
    \item{Equilateral Triangles \\}
    
      $V=\int_{a}^{b}{\frac{\sqrt{3}(2s)^{2}}{4}}\,dx=\frac{\sqrt{3}I}{4}\approx 18.475$ \\
      
    \item{Semicircles \\} 
    
      $V=\int_{a}^{b}{\frac{1}{2}\pi r^{2}}\,dx=\frac{\pi}{2}\int_{-2}^{2}{\left(\frac{2\sqrt{4-x^{2}}}{2}\right)^{2}}\,dx=\frac{I\pi}{8}\approx 16.755$ \\
      
    \item{Isosceles right triangles with the hypotenuse as the base of the solid \\}
    
      $V=\int_{a}^{b}{\frac{1}{2}(2s)^{2}}\,dx=\frac{I}{2}\approx 21.333$ \\
      
  \end{enumerate}
  
  \pagebreak
  
  \item{The base of a solid is bounded by $y=x^{3}$, $y=0$, and $x=1$. Find the volume of the solid for each of the following cross sections taken perpendicular to the $y$-axis. \\}
  \begin{multicols}{2}
  
    $y=x^{3}\rightarrow x=\sqrt[3]{y}$ \\
    
    $x=1\rightarrow y=x^{3}=1$ \\
    
    Bounds: $a=0$, $b=1$ \\
    
    $f(x)-g(x)=1-\sqrt[3]{y}$ \\
    
    \vfill\null
    
    \columnbreak
    
    \begin{center}
      \includegraphics[scale=0.22]{7.2_q3.png} \\
    \end{center} \\
    \columnbreak
  \end{multicols}
  \begin{enumerate}
    \item{Squares \\}
    
      $V=\int_{a}^{b}{s^{2}}\,dy=\int_{0}^{1}{\left(1-\sqrt[3]{y}\right)^{2}}\,dy=[y-\frac{3}{2}y^{\frac{4}{3}}+\frac{3}{5}y^{\frac{5}{3}}]_{0}^{1}=1-\frac{3}{2}+\frac{3}{5}=\frac{1}{10}=0.1$\\
      
      Going forward, let $I=\int_{a}^{b}{s^{2}}\,dy=\int_{0}^{1}{\left(1-\sqrt[3]{y}\right)^{2}}\,dy=0.1$. \\
      
    \item{Semicircles \\}
    
      $V=\int_{a}^{b}{\frac{1}{2}\pi r^{2}}\,dy=\frac{\pi}{2}\int_{0}^{1}{\left(\frac{1-\sqrt[3]{y}}{2}\right)^{2}}=\frac{I\pi}{8}=0.039$ \\
      
    \item{Equilateral Triangles \\}
    
      $V=\int_{a}^{b}{\frac{\sqrt{3}s^{2}}{4}}=\frac{\sqrt{3}I}{4}=0.043$ \\
      
    \item{Semiellipses whose heights are twice the length of their bases \\}
    
      $V=\int_{a}^{b}{\frac{1}{2}\pi ab}\,dy=\frac{\pi}{2}\int_{0}^{1}{\left(\frac{1-\sqrt[3]{y}}{2}\right)\left(2\left(1-\sqrt[3]{y}\right)\right)}\,dy=\frac{I\pi}{2}\approx 0.157$ \\
      
  \end{enumerate}
\end{enumerate}
\end{document}
