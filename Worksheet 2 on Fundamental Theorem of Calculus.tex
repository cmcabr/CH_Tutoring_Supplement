\documentclass[10pt, letterpaper]{report}
\usepackage[letterpaper]{geometry}
  \geometry{top=1in, bottom=1in, left=1in, right=1in}
\usepackage[utf8]{inputenc}
\usepackage{textcomp, gensymb, mathtools , amssymb , amsthm, graphicx}
  \graphicspath{ {/home/lowebang/Pictures/} }
\usepackage{fancyhdr}
  \pagestyle{fancy}
  \lhead{}
  \chead{}
  \rhead{Cabrera \thepage}
  \lfoot{}
  \cfoot{}
  \rfoot{\LaTeX}
  \renewcommand{\headrulewidth}{1pt}
  \renewcommand{\footrulewidth}{1pt}
  \setlength\headsep{0.333in}

\title{Calculus BC - Worksheet 2 on Fundamental Theorem of Calculus}
\author{Craig Cabrera}
\date{6 January 2022}

\begin{document}
\maketitle
Work the following on \textbf{\underline{notebook paper}}. Use your calculator on problems 3, 8, and 13.
\begin{enumerate}
	\item{If $f(1)=12$, $f'$ is continuous, and $\int_{1}^{4}{f'(x)}\,dx=17$, what is the value of $f(4)$? \\}

		$f'(x)$ is continuous on $[1,4] \therefore f(x)$ is continuous on $[1,4]$ and differentiable on $(1,4)$ (Fundamental Theorem of Calculus holds.) \\

		$\int_{1}^{4}{f'(x)}\,dx=
		f(4)-f(1)=17\rightarrow
		f(4)=17+f(1)=17+12=29 \\$

	\item{If $\int_{2}^{5}{2f(x)+3}\,dx=17$, find $\int_{2}^{5}{f(x)}\,dx$. \\}

    No test of differentiablity is used as all integrals are either simplified into variables or are integrals of continuous constants. \\

		Let $F=\int_{2}^{5}{f(x)}\,dx \\$

		$2F+3\int_{2}^{5}\,dx=
		2F+3[x]_{2}^{5}=
		2F+3[3]=17\rightarrow
		2F=8\rightarrow F=\int_{2}^{5}{f(x)}\,dx=4$ \\

	\item{Water is pumped out of a holding tank at a rate of $5-5e^{-0.12t}$ litres/minute, where $t$ is in minutes since the pump is started. If the holding tank contains 1000 litres of water when the pump is started, how much water does it hold one hour later? \\}

		Test of Differentiability:
		$\lim_{t\to 0}r(t)=r(0)=0,
		\lim_{t\to 60}r(t)=r(60)\approx4.996 \\$
		$\therefore R(t)$ is continuous on $[0,60]$ and differentiable on $(0,60)$ (Fundamental Theorem of Calculus holds.) \\

		$5\int_{0}^{60}\,dt-5\int_{0}^{60}{e^{-0.12t}}\,dt=
		5[x]_{0}^{60}-5[-\frac{1}{0.12}e^{-0.12t}]_{0}^{60}\approx
		5[60]-5[-0.006+8.333]=300-41.635=258.365$ litres pumped out of a tank in one hour. \\

		$1000-258.364=741.636$ litres in the tank one hour later. \\
	\item{Given the values of the derivative $f'(x)$ in the table and that $f(0)=100$, estimate $f(x)$ for $x=2,4,6$. Use a right Riemann sum.
		\begin{center}
			\begin{tabular}{| c | c | c | c | c |}
				\hline
					$x$ & 0 & 2 & 4 & 6 \\
				\hline
					$f'(x)$ & 10 & 18 & 23 & 25 \\
				\hline
			\end{tabular}
		\end{center}}

    $f'(x)$ is continuous on $[0,6] \therefore f(x)$ is continuous on $[0,6]$ and differentiable on $(0,6)$ (Fundamental Theorem of Calculus holds.) \\

    \begin{itemize}
      \item{$\sum_{i=1}^{2}{f(x_{i}^{*})}=2(18)=36 \\$

        $\int_{0}^{2}{f'(x)}\,dx=
        f(2)-f(0)=36\rightarrow
        f(2)=36+f(0)=36+100=136$}
      \item{$\sum_{i=1}^{2}{f(x_{i}^{*})}=2(23)=46 \\$

        $\int_{2}^{4}{f'(x)}\,dx=
        f(4)-f(2)=46\rightarrow
        f(4)=136+f(2)=136+46=182$}
      \item{$\sum_{i=1}^{2}{f(x_{i}^{*})}=2(25)=50 \\$

        $\int_{4}^{6}{f'(x)}\,dx=
        f(6)-f(4)=50\rightarrow
        f(6)=50+f(4)=50+182=232$}
    \end{itemize}
\pagebreak
	\item{Consider the function $f$ that is continuous on the interval $[-5,5]$ and for which $\int_{0}^{5}{f(x)}\,dx=4$. \\ Evaluate:\\}

    $f'(x)$ is continuous on $[-5,5] \therefore f(x)$ is continuous on $[-5,5]$ and differentiable on $(-5,5)$ for all \\ situations (Fundamental Theorem of Calculus holds.) \\

		\begin{enumerate}
			\item{$\int_{0}^{5}{f(x)+2}\,dx=$\\}

				$\int_{0}^{5}{f(x)}\,dx+2\int_{0}^{5}\,dx=
        \int_{0}^{5}{f(x)}+2[x]_{0}^{5}=
        4+10=14\\$

			\item{$\int_{-2}^{3}{f(x+2)}\,dx=$} \\

				Let $u=x+2\therefore du=dx$ \\

        $\int_{0}^{5}{f(x)}\,dx=4$ \\

			\item{$\int_{-5}^{5}{f(x)}\,dx$ ($f$ is even)$=$} \\

				Because an even function has reflection symmetry about the y-axis,
        $\int_{-5}^{0}{f(x)}\,dx=
        \int_{0}^{5}{f(x)}\,dx$ \\

        $\int_{-5}^{5}{f(x)}\,dx=
        \int_{-5}^{0}{f(x)}\,dx+\int_{0}^{5}{f(x)}\,dx=
        4+4=8$ \\

			\item{$\int_{-5}^{5}{f(x)}\,dx$ ($f$ is odd)$=$} \\

				Because an odd function has rotational symmetry about the origin,
        $\int_{-5}^{0}{f(x)}\,dx=
        -\int_{0}^{5}{f(x)}\,dx$ \\

        $\int_{-5}^{5}{f(x)}\,dx=
        \int_{-5}^{0}{f(x)}\,dx+\int_{0}^{5}{f(x)}\,dx=
        4+(-4)=0$ \\

		\end{enumerate}
	\item{Use the figure on the first page and the fact that $P(0)=2$ to find values of $P$ when $t=1,2,3,4,$ and 5. \\}

    $\frac{dP}{dt}$ is continuous on $[0,5] \therefore P(t)$ is continuous on $[0,5]$ and differentiable on $(0,5)$ (Fundamental Theorem of Calculus holds. ) \\

		\begin{itemize}
      \item{$\int_{0}^{1}{\frac{dP}{dt}}\,dt\approx
        bh=(1)(-1)=
        P(1)-P(0)=-1\rightarrow
        P(1)=-1+P(0)=-1+2=1$ \\}
      \item{$\int_{1}^{2}{\frac{dP}{dt}}\,dt\approx
        bh=(1)(-1)=
        P(2)-P(1)=-1\rightarrow
        P(2)=-1+P(1)=-1+1=0$ \\}
      \item{$\int_{2}^{3}{\frac{dP}{dt}}\,dt\approx
        \frac{bh}{2}=\frac{(1)(-1)}{2}=
        P(3)-P(2)=-\frac{1}{2}\rightarrow
        P(3)=P(2)-\frac{1}{2}=-\frac{1}{2}$ \\}
      \item{$\int_{3}^{4}{\frac{dP}{dt}}\,dt\approx
        \frac{bh}{2}=\frac{1*1}{2}=
        P(4)-P(3)=-\frac{1}{2}\rightarrow
        P(4)=P(3)-\frac{1}{2}=\frac{1-1}{2}=0$ \\}
      \item{$\int_{4}^{5}{\frac{dP}{dt}}\,dt\approx
        bh=(1)(1)=
        P(5)-P(4)=1\rightarrow
        P(5)=1-P(4)=1$ \\}
    \end{itemize}
\pagebreak
	\item{In the figure on the first page, the graph of $g$ is given. Let $G(t)$ be the antiderivative of $g(t)$.\\}

    $g(t)$ is continuous on $[0,5] \therefore G(t)$ is continuous on $[0,5]$ and differentiable on $(0,5)$ (Fundamental Theorem of Calculus holds.) \\
		\begin{enumerate}
			\item{Given $G(0)=5$, find $G(2)$, $G(4)$, and $G(5)$\\}
				\begin{itemize}
          \item{$\int_{0}^{2}{g(t)}\,dt=
            G(2)-G(0)=16\rightarrow
            G(2)=16+G(0)=16+5=21$\\}
          \item{$\int_{2}^{4}{g(t)}\,dt=
            G(4)-G(2)=-8\rightarrow
            G(4)=G(2)-8=21-8=13$\\}
          \item{$\int_{4}^{5}{g(t)}\,dt=
            G(5)-G(4)=2\rightarrow
            G(5)=2+G(4)=2+13=15$\\}
        \end{itemize}
			\item{Find the intervals where the graph of $G$ is increasing and decreasing. Justify your answer.\\}
				\begin{itemize}
          \item{$g(t)$ is positive on $0<t<2\cup4<t<5\therefore G(t)$ is increasing on $0<t<2\cup4<t<5$. \\}
          \item{$g(t)$ is negative on $2\leq t\leq 4\therefore G(t)$ is decreasing on $2\leq t\leq 4$.\\}
        \end{itemize}
			\item{Find the intervals where the graph of $G$ is concave up and concave down. Justify your answer.\\}
				\begin{itemize}
          \item{$g(t)$ is increasing on $0<t<\frac{2}{3}\cup 3<x<\frac{9}{2}\therefore G(t)$ is concave up on $0<t<\frac{2}{3}\cup 3<x<\frac{9}{2}$. \\}
          \item{$g(t)$ is decreasing on $\frac{2}{3}<t<3\cup4.5<t<5\therefore G(t)$ is concave down on $\frac{2}{3}<t<3\cup4.5<t<5$.\\}
        \end{itemize}
			\item{Sketch a graph of an antiderivate $G(t)$. Label each critical point of $G(t)$ with its coordinates.\\}

        Graph on next page. \\

		\end{enumerate}
	\item{Find the value of $F(1)$, where $F'(1)=e^{-x^{2}}$ and $F(0)=2$.\\}

    Test of Differentiability:
    $\lim_{x \to 0}F'(x)=F'(0)=1,
     \lim_{x \to 1}F'(x)=F'(1)\approx0.368\\$
     $\therefore F(x)$ is continuous on $[0,1]$ and differentiable on $(0,1)$ (Fundamental Theorem of Calculus holds.) \\


		$\int_{0}^{1}{F'(x)}\,dx=
      F(1)-F(0)\approx0.747\rightarrow
      F(1)=0.747+F(0)=2.747$

	\item{Given $f(x)=$$\begin{cases}
		2x, x\leq 1 \\
		2, x>1
		\end{cases}$. Evaluate $\int_{\frac{1}{2}}^{5}{f(x)}\,dx$.\\}

    $f(x)$ is continuous on $[\frac{1}{2},5] \therefore F(x)$ is continuous on $[\frac{1}{2},5]$ and differentiable on $(\frac{1}{2},5)$ (Fundamental Theorem of Calculus holds.) \\

		$\int_{\frac{1}{2}}^{5}{f(x)}\,dx=
    [2x]_{x=5}-[x^{2}]_{x=\frac{1}{2}}=
    10-\frac{1}{4}=\frac{35}{4}=8.75 \\$
\pagebreak
	\item{A bowl of soup is placed on a kitchen counter to cool. The temperature of the soup is given in the table below.
		\begin{center}
			\begin{tabular}{| c | c | c | c | c |}
				\hline
					Time $t$ (minutes) & 0 & 5 & 8 & 12 \\
				\hline
					Temperature $T(x)$ (\degree F) & 105 & 99 & 97 & 93 \\
				\hline
			\end{tabular}
		\end{center}}

    Test of Differentiability:
    No $x$ or $y$-value are the same $\therefore \forall\, x \exists\, T'(x) \neq0$ \\
    $\therefore T(x)$ is continuous on $[0.12]$ and differentiable on $(0,12)$ (Fundamental Theorem of Calculus holds.) \\

		\begin{enumerate}
			\item{Find $\int_{0}^{12}{T'(x)}\,dx$.} \\

				$\int_{0}^{12}{T'(x)}\,dx=
        [T(x)]_{0}^{12}=
        T(12)-T(0)=93-105=-12$\degree F \\

			\item{Find the average rate of change of $T(x)$ over the time interval $t=5$ to $t=8$ minutes.} \\

        Average Value Function:
        \[ \Delta x=\frac{b-a}{n}\rightarrow
        n=\frac{b-a}{\Delta x}\]

        \[f_{avg}={\frac{\lim_{n\to \infty}\sum_{i=1}^{n}f(x_{i}^{*})}{n}}=
        {\frac{\lim_{n\to \infty}\sum_{i=1}^{n}f(x_{i}^{*})\Delta x}{b-a}=
         \frac{\int_{a}^{b}{f(x)}\,dx}{b-a}} \]

				$T_{avg}=\frac{\int_{5}^{8}{T'(x)}\,dx}{8-5}=
        \frac{T(8)-T(5)}{8-5}=
        \frac{97-99}{8-5}=-\frac{2}{3}$ \degree F/minute.

		\end{enumerate}
	\item{The graph of $f'$ which consists of a line segment and a semicircle, is shown on the second page. Given that $f(1)=4$, find:\\}

    $f'(x)$ is continuous on $[-2,5]\therefore f(x)$ is continuous on $[-2,5]$ and differentiable on $(-2,5)$ (Fundamental Theorem of Calculus holds.) \\

		\begin{enumerate}
			\item{$f(-2)$\\}

				$\int_{-2}^{1}{f'(x)}\,dx\approx
        \sum_{i=1}^{2}f(x)\Delta x=
        \frac{-4*2}{2}+\frac{1*2}{2}=
        1-4=-3\\$

        $f(1)-f(-2)=-3\rightarrow
        f(-2)=3+f(1)=3+4=7\\$

			\item{$f(5)$\\}

				$\int_{1}^{5}{f'(x)}\,dx\approx
        \sum_{i=1}^{2}f(x)\Delta x=
        bh-\frac{\pi*r^2}{2}=
        (4*2)-\frac{4\pi}{2}=
        8-2\pi\approx1.717\\$

        $f(5)-f(1)=8-2\pi\rightarrow
        f(5)=f(1)+8-2\pi=12-2\pi\approx5.717\\$

		\end{enumerate}
\pagebreak
	\item{(Multiple Choice) If $f$ and $g$ are continuous functions such that $g'(x)=f(x)$ for all $x$, then $\int_{2}^{3}{f(x)}\,dx=$\\}
		\begin{enumerate}
			\item{$g'(2)-g'(3)$}
			\item{$g'(3)-g'(2)$}
			\item{$g(3)-g(2)$}
			\item{$f(3)-f(2)$}
			\item{$f'(3)-f'(2)$\\}
		\end{enumerate}

    $f$ and $g$ are continuous functions such that $g'(x)=f(x)$ for all $x\therefore F(x)=g(x)$ is continuous and differentiable on $(-\infty, \infty)$ (Fundamental Theorem of Calculus holds.) \\

    $\int_{2}^{3}{f(x)}\,dx=
    F(3)-F(2)=g(3)-g(2)\\$
	\item{(Multiple Choice) If the function $f(x)$ is defined by $f(x)=\sqrt{x^{3}+2}$ and $g$ is an antiderivate of $f$ such that $g(3)=5$, then $g(1)=$\\}
		\begin{enumerate}
			\item{$-3.268$}
			\item{$-1.585$}
			\item{$1.732$}
			\item{$6.585$}
			\item{$11.585$\\}
		\end{enumerate}

    Test of Differentiability:
    $\lim_{x\to 1}f(x)=f(1)=\sqrt{3}\approx1.732,
    \lim_{x\to 3}f(x)=f(3)=\sqrt{29}\approx5.385$\\
    $\therefore g(x)$ is continuous on $[1,3]$ and differentiable on $(1,3)$ (Fundamental Theorem of Calculus holds.) \\

    $\int_{1}^{3}{\sqrt{x^{3}+2}}\,dx\approx6.585\\$

    $g(3)-g(1)=5-g(1)=6.585\rightarrow
    g(1)=5-6.585=-1.585\\$

	\item{(Multiple Choice) The graph of $f$ is shown in the figure on the second page. If $\int_{1}^{3}{f(x)}\,dx=2.3$ and $F'(x)=f(x)$, then $F(3)-F(0)=$} \\
		\begin{enumerate}
			\item{$0.3$}
			\item{$1.3$}
			\item{$3.3$}
			\item{$4.3$}
			\item{$5.3$\\}
		\end{enumerate}

    $f(x)$ is continuous on $[1,3]\therefore F(x)$ is continuous on $[1,3]$ and differentiable on $(1,3)$ (Fundamental Theorem of Calculus holds.) \\

		$F(3)-F(0)=
    \int_{0}^{3}{f(x)}\,dx=
    \int_{0}^{1}{f(x)}\,dx+\int_{1}^{3}{f(x)}\,dx=
    bh+2.3=
    (2*1)+2.3=4.3\\$

\end{enumerate}
\end{document}
