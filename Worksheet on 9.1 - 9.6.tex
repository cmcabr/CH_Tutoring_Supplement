% Document Metadata
\documentclass[10pt,letterpaper]{report}
\usepackage[utf8]{inputenc}
% Use for Arial Font \usepackage{helvet}
%  \renewcommand{\familydefault}{\sfdefault}
% Use for Times New Roman Font \usepackage{mathptmx}

\usepackage[none]{hyphenat}
\usepackage{tikz, textcomp, gensymb, graphicx, mathtools, amssymb, amsthm, hyperref, multicol}
  \hypersetup{
      colorlinks=true,
      linkcolor=blue,
      filecolor=magenta,
      urlcolor=blue,
      }
  \graphicspath{ {/home/lowebang/Pictures/} }
\usepackage[letterpaper]{geometry}
  \geometry{top=1in, bottom=1in, left=1in, right=1in}
\usepackage{fancyhdr}
  \pagestyle{fancy}
  \lhead{}
  \chead{}
  \rhead{Cabrera \thepage}
  \lfoot{}
  \cfoot{}
  \rfoot{\LaTeX}
  \renewcommand{\headrulewidth}{1pt}
  \renewcommand{\footrulewidth}{1pt}
  \setlength\headsep{0.333in}

% Command to Circle String
\newcommand*\circled[1]{\tikz[baseline=(char.base)]{
            \node[shape=circle,draw,inner sep=2pt] (char) {#1};}}

% Command to Set Oval Around String
\newcommand{\mymk}[1]{%
  \tikz[baseline=(char.base)]\node[anchor=south west, draw,rectangle, rounded corners, inner sep=2pt, minimum size=7mm,
  text height=2mm](char){\ensuremath{#1}} ;}

\title{Calculus BC -- Worksheet on 9.1 - 9.6 }
\author{Craig Cabrera}
\date{21 March 2022}

\begin{document}
\maketitle
\begin{center}
  \textbf{\underline{Relevant Formulas and Notes:}}
\end{center}

\noindent Ratio Test: \\

\noindent Suppose we have the series $\sum_{n=0}^{\infty}{a_{n}}$. Consider: \\

$$\lim_{n\to\infty}{\left|\frac{a_{n+1}}{a_{n}}\right|}$$ \\

\noindent Then, the series is convergent if $\lim_{n\to\infty}{\left|\frac{a_{n+1}}{a_{n}}\right|} < 1$. It is divergent if the limit is greater than one, and must be proven with another method if the limit equals one. \\

\noindent Root Test: \\

\noindent Suppose we have the series $\sum_{n=0}^{\infty}{a_{n}}$. Consider:

$$\lim_{n\to\infty}{\sqrt[n]{\left|a_{n}\right|}}=\lim_{n\to\infty}{\left|a_{n}\right|^{\frac{1}{n}}}$$ \\

\noindent Then, the series is convergent if $\lim_{n\to\infty}{\left|a_{n}\right|^{\frac{1}{n}}} < 1$. It is divergent if the limit is greater than one, and must be proven with another method if the limit equals one. \\

\pagebreak


\noindent Work the following on \textbf{\underline{notebook paper}}.
\noindent Use the Ratio Test to determine the convergence or divergence of the series.
\begin{enumerate}
  \item{$\sum_{n=0}^{\infty}{\frac{n!}{3^{n}}}$ \\}

    Let $a_{n}=\frac{n!}{3^{n}}$ and $a_{n+1}=\frac{(n+1)!}{3^{n+1}}$ \\

    $\lim_{n\to\infty}{\left|\frac{a_{n+1}}{a_{n}}\right|}=
    \lim_{n\to\infty}{\left|\frac{\frac{(n+1)!}{3^{n+1}}}{\frac{n!}{3^{n}}}\right|}=
    \lim_{n\to\infty}{\left|\frac{3^{n}(n+1)!}{3^{n+1}n!}\right|}=
    \lim_{n\to\infty}{\left|\frac{n+1}{3}\right|}=\infty$ \\

    $\therefore \sum_{n=0}^{\infty}{\frac{n!}{3^{n}}}$ diverges by the Ratio Test. \\

  \item{$\sum_{n=1}^{\infty}{\frac{n}{4^{n}}}$ \\}

    Let $a_{n}=\frac{n}{4^{n}}$ and $a_{n+1}=\frac{n+1}{4^{n+1}}$ \\

    $\lim_{n\to\infty}{\left|\frac{a_{n+1}}{a_{n}}\right|}=
    \lim_{n\to\infty}{\left|\frac{\frac{n+1}{4^{n+1}}}{\frac{n}{4^{n}}}\right|}=
    \lim_{n\to\infty}{\left|\frac{4^{n}(n+1)}{n4^{n+1}}\right|}=
    \lim_{n\to\infty}{\left|\frac{n+1}{4n}\right|}=
    \lim_{n\to\infty}{\left|\frac{1+\frac{1}{n}}{4}\right|}=\frac{1}{4}$ \\

    $\therefore \sum_{n=1}^{\infty}{\frac{n}{4^{n}}}$ converges by the Ratio Test. \\

  \item{$\sum_{n=0}^{\infty}{\frac{(-1)^{n}2^{n}}{n!}}$ \\}

    Let $a_{n}=\frac{(-1)^{n}2^{n}}{n!}$ and $a_{n+1}=\frac{(-1)^{n+1}2^{n+1}}{(n+1)!}$ \\

    $\lim_{n\to\infty}{\left|\frac{a_{n+1}}{a_{n}}\right|}=
    \lim_{n\to\infty}{\left|\frac{\frac{(-1)^{n+1}2^{n+1}}{(n+1)!}}{\frac{(-1)^{n}2^{n}}{n!}}\right|}=
    \lim_{n\to\infty}{\left|\frac{(-1)^{n+1}2^{n+1}n!}{(-1)^{n}2^{n}(n+1)!}\right|}=
    \lim_{n\to\infty}{\left|-\frac{2}{n+1}\right|}=0$ \\

    $\therefore \sum_{n=0}^{\infty}{\frac{(-1)^{n}2^{n}}{n!}}$ converges by the Ratio Test. \\

    \hline


\noindent Use the Root Test to determine the convergence or divergence of the series. \\

  \item{$\sum_{n=1}^{\infty}{\left(\frac{n}{2n+1}\right)^n}$ \\}

    $\lim_{n\to\infty}{\sqrt[n]{\left|a_{n}\right|}}=
    \lim_{n\to\infty}{\sqrt[n]{\left|\left(\frac{n}{2n+1}\right)^n\right|}}=
    \lim_{n\to\infty}{\left|\frac{n}{2n+1}\right|}=
    \lim_{n\to\infty}{\left|\frac{1}{2+\frac{1}{n}}\right|}=\frac{1}{2}$ \\

    $\therefore \sum_{n=1}^{\infty}{\left(\frac{n}{2n+1}\right)^n}$ converges by the Root Test. \\

  \item{$\sum_{n=2}^{\infty}{\frac{(-1)^{n}}{(\ln{n})^{n}}}$ \\}

    $\lim_{n\to\infty}{\sqrt[n]{\left|a_{n}\right|}}=
    \lim_{n\to\infty}{\sqrt[n]{\left|\frac{(-1)^{n}}{(\ln{n})^{n}}\right|}}=
    \lim_{n\to\infty}{\left|-\frac{1}{\ln{n}}\right|}=0$ \\

    $\therefore \sum_{n=2}^{\infty}{\frac{(-1)^{n}}{(\ln{n})^{n}}}$ converges by the Root Test. \\

    \pagebreak

\noindent Determine whether each of the given \textbf{\underline{series}} converges or diverges/ You must use each of the ten tests at least once. Show all work, and \textbf{\underline{justify}} your answers. \\

  \item{$\sum_{n=1}^{\infty}{\frac{1}{n^{\frac{3}{2}}}}$ \\}

    $ $ \\

  \item{}

\end{enumerate}
\end{document}
