\documentclass[10pt, letterpaper]{report}
\usepackage[letterpaper]{geometry}
  \geometry{top=1in, bottom=1in, left=1in, right=1in}
\usepackage[utf8]{inputenc}
\usepackage{textcomp, gensymb, mathtools , amssymb , amsthm, graphicx}
  \graphicspath{ {/home/lowebang/Pictures/} }
\usepackage{fancyhdr}
  \pagestyle{fancy}
  \lhead{}
  \chead{}
  \rhead{Cabrera \thepage}
  \lfoot{}
  \cfoot{}
  \rfoot{\LaTeX}
  \renewcommand{\headrulewidth}{1pt}
  \renewcommand{\footrulewidth}{1pt}
  \setlength\headsep{0.333in}

\title{Calculus BC - Worksheet on Average Value}
\author{Craig Cabrera}
\date{13 January 2022}

\begin{document}
\maketitle
\noindent Work the following on \textbf{\underline{notebook paper}}. Use your calculator on problems 3-6, and give decimal answers correct to \textbf{\underline{three}} decimal places. \\

\noindent Mean Value Theorem for Integrals: If the Fundamental Theorem of Calculus holds (i.e. $F(x)$ is differentiable on $(a,b)$ or $f(x)$ is continuous on $[a,b]$), then there exists some value $c$ such that: \\

\[f(c)=\frac{F(b)-F(a)}{b-a}=\frac{\int_{a}^{b}{f(x)}}{b-a}\,dx\rightarrow\int_{a}^{b}{f(x)}\,dx=f(c)(b-a)\]

\begin{enumerate}
  \item{$f(x)=(x-3)^{2}, [2,5]$} \\

    Test of Continuity:
    $\lim_{x \to 2}f(x)=f(2)=1,
    \lim_{x \to 5}f(x)=f(5)=4\\$
    $\therefore f(x)$ is continuous on $[2,5]$ (Fundamental Theorem of Calculus holds.) \\

    \begin{enumerate}
      \item{Find the average value of $f$ on the given interval.} \\

        $f_{AVE}=\frac{\int_{2}^{5}{f(x)}\,dx}{5-2}=
          \frac{\int_{2}^{5}{x^{2}}\,dx-6\int_{2}^{5}{x}\,dx+9\int_{2}^{5}\,dx}{5-2}=
          \frac{[\frac{125-8}{3}]-6[\frac{25-4}{2}]+9[5-2]}{5-2}=
          \frac{39-63+27}{3}=13-21+9=1$ \\

      \item{Find the value of $c$ such that $f_{AVE}=f(c)$} \\

        $\int_{2}^{5}{f(x)}\,dx=1(5-2)=3\therefore f(c)=1$ \\

        $(x-3)^{2}=x^{2}-6x+9=1\rightarrow
        x^2-6x+8=(x-2)(x-4)=0\therefore x=2,4$ \\

    \end{enumerate}
  \item{$f(x)=\sqrt{x}, [0,4]$} \\

    Test of Continuity:
    $\lim_{x \to 0}f(x)=f(0)=0,
    \lim_{x \to 4}f(x)=f(4)=2\\$
    $\therefore f(x)$ is continuous on $[0,4]$ (Fundamental Theorem of Calculus holds.) \\
    \begin{enumerate}
      \item{Find the average value of $f$ on the given interval.} \\

        $f_{AVG}=\frac{\int_{0}^{4}{f(x)}\,dx}{4-0}=
        \frac{[\frac{2}{3}x^{\frac{3}{2}}]_{0}^{4}}{4}=
        \frac{[\frac{16}{3}]}{4}=\frac{4}{3}\approx1.333$ \\

      \item{Find the value of $c$ such that $f_{AVE}=f_{c}$} \\

        $\int_{2}^{5}{f(x)}\,dx=\frac{4}{3}(4-0)=\frac{16}{3}\therefore f(c)=\frac{4}{3}$ \\

        $\sqrt{x}=\frac{4}{3}\rightarrow x=\frac{4^{2}}{3^{2}}=\frac{16}{9}\approx1.778$ \\

    \end{enumerate}
\pagebreak
  \item{The table below gives values of a continuous function. Use a midpoint Riemann Sum with three equal subintervals to estimate the average value of $f$ on $[20,50]$. \\
    \begin{center}
      \begin{tabular}{| c | c | c | c | c | c | c | c |}
        \hline
          $x$ & 20 & 25 & 30 & 35 & 40 & 45 & 50 \\
        \hline
          $f(x)$ & 42 & 38 & 31 & 29 & 35 & 48 & 60 \\
        \hline
      \end{tabular}
    \end{center}}

    Test of Continuity:
    All values listed exist within the function \\
    $\therefore f(x)$ is continuous on $[0.12]$ and differentiable on $(0,12)$ (Fundamental Theorem of Calculus holds.) \\

    $f_{avg}\approx
    \frac{\sum_{i=1}^{3}{10f(x)}}{50-20}=
    \frac{10(38+29+48)}{30}=
    \frac{1150}{30}=\frac{115}{3}\approx38.333$ \\
\hline
  \item{The velocity graph of an accelerating car is shown on the first page.} \\

    $v(t)$ is continuous on $[0,12]$ (Fundamental Theorem of Calculus holds.)

    \begin{enumerate}
      \item{Estimate the average velocity of the car during the first 12 seconds by using a midpoint Riemann sum with three equal subintervals.} \\

        $v_{avg}\approx
        \frac{\sum_{i=1}^{3}{4v(t)}}{12-0}=
        \frac{4(20+50+65)}{12}=
        \frac{540}{12}=
        \frac{135}{3}=45$ kilometres/hour on average \\

      \item{At what time was the instantaneous velocity equal to the average velocity?} \\

        $\int_{0}^{12}{v(t)}\,dt=
        45(12-0)\approx540$ km $\therefore f(c)=45$ kilometres/hour on average \\

        $v(t)=45$ km/hr when $t\approx5$ seconds \\

        (Graph is a bit confusing on units; $x$-axis labeled in seconds whereas $y$-axis is labeled in kilometres per hour.)

    \end{enumerate}
\hline
  \item{In a certain city, the temperature, in \degree F, $t$ hours after 9 AM was modeled by the function $T(t)=50+14\sin{(\frac{\pi\,t}{12})}$. Find the average temperature during the period from 9 AM to 9 PM. } \\

    Test of Continuity:

    $\lim_{t\to 0}T(t)=T(0)=\lim_{t\to 12}T(t)=T(12)=50$
    $\therefore T(t)$ is continuous on $[0,12]$ (Fundamental Theorem of Calculus holds.) \\

    $T_{avg}=\frac{\int_{0}^{12}{T(t)}\,dt}{12-0}=
    \frac{50\int_{0}^{12}\,dt+14\int_{0}^{12}{\sin{(\frac{\pi\,t}{12})}}\,dt}{12}=
    \frac{50[x]_{0}^{12}+14[-\frac{12}{\pi}\cos{(\frac{\pi\,t}{12})}]_{0}^{12}}{12}=
    \frac{50[12]-\frac{168}{\pi}[-2]}{12}=
    \frac{50[12]-\frac{168}{\pi}[-2]}{12}=$ \\

    $\frac{600+\frac{336}{\pi}}{12}=
    50+\frac{28}{\pi}\approx 58.913$ \degree F on average \\
\pagebreak
  \item{If a cup of coffee has temperature 95\degree C in a room where the temperature is 20\degree C, then, according to Newton's Law of Cooling, the temperature of the coffee after $t$ minutes is given by the function $T(t)=20+75e^{-\frac{t}{50}}$. What is the average temperature of the coffee during the first half hour?} \\

    Test of Continuity:

    $\lim_{t\to 0}T(t)=T(0)=95,
    \lim_{t\to 30}T(t)=T(30)\approx61.161$ \\
    $\therefore T(t)$ is continuous on $[0,30]$ (Fundamental Theorem of Calculus holds.) \\

    $f_{avg}=\frac{\int_{0}^{30}{T'(t)}\,dt}{30-0}=
    \frac{20\int_{0}^{30}\,dt+75\int_{0}^{30}{e^{-\frac{t}{50}}}}{30}=
    \frac{20[x]_{0}^{30}+75[-50e^{-\frac{t}{50}}]_{0}^{30}}{30}\approx
    \frac{20[30]-3750[-0.451]}{30}=
    20-125(-0.451)=$ \\

    $20+56.399=76.399$ \degree C, on average. \\

\hline
  \item{Suppose the $C(t)$ represents the daily cost of heating your house, measured in dollars per day, where $t$ is time measured in days and $t=0$ corresponds to January 1, 2010. Interpret $\int_{0}^{90}{C(t)}\,dt$ and $\frac{1}{90-0}\int_{0}^{90}{C(t)}\,dt$.} \\
    \begin{itemize}
      \item{$\int_{0}^{90}{C(t)}\,dt$ is an expression representing the estimated total cost in dollars of heating your home 90 days into 2010 (or in other terms, from 1 January 2010 to 1 April 2010).} \\

      \item{$\frac{1}{90-0}\int_{0}^{90}{C(t)}\,dt$ is an expression representing the estimated average cost per day of heating your home 90 days into 2010 (or in other terms, from 1 January 2010 to 1 April 2010).} \\
    \end{itemize}
  \item{Using the figure on the second page,} \\

  $f(x)$ is continuous on $[1,6]$. It does not need to be differentiable as we are integrating the function. \\
    \begin{enumerate}
      \item{Find $\int_{1}^{6}{f(x)}\,dx$.} \\

        $\int_{1}^{6}{f(x)}\,dx\approx
        \sum_{i=1}^{4}{f(x)}\Delta x=
        bh+bh+\frac{bh}{2}+\frac{bh}{2}=
        \frac{2*1}{2}+\frac{1*1}{2}+5+2=
        1+\frac{1}{2}+5+2=8.5$ \\

      \item{What is the average value of $f$ on $[1,6]$?} \\

        $f_{avg}=\frac{\int_{1}^{6}{f(x)}\,dx}{6-1}=
        \frac{8.5}{5}=1.7$ \\
    \end{enumerate}
\hline
  \item{The average value of $y=f(x)$ equals 4 for $1\leq x\leq 6$ and equals 5 for $6\leq x\leq 8$. \\

	What is the average value of $f(x)$ for $1\leq x\leq 8$?} \\

    Because average values for each interval are valid, we can assume the Fundamental Theorem of \\ Calculus holds. \\

    $\int_{1}^{8}{f(x)}\,dx=
    \int_{1}^{6}{f(x)}\,dx+\int_{8}^{6}{f(x)}\,dx=
    4*(6-1)+5*(8-6)=10=30$ \\

    $f_{avg}=\frac{\int_{1}^{8}{f(x)}\,dx}{8-1}=\frac{30}{7}$ \\

\pagebreak
  \par In problems 10-11, find the average value of the function on the given integral without integrating.
  \par Hint: Use Geometry. (No calculator)
  \item{$f(x)=\begin{cases}
    	x+4, -4\leq x\leq -1 \\
        -x+2, -1\leq x\leq 2
    \end{cases}$ on $[-4,2]$} \\

    Test of Continuity:

    $\lim_{x\to -4}f(x)=f(-4)=0,
    \lim_{x\to -1}f(x)=f(-1)=3,
    \lim_{x\to 2}f(x)=f(2)=0$ \\
    $\therefore f(x)$ is continuous on $[-4,2]$ (Fundamental Theorem of Calculus holds.)

    $f_{avg}=\frac{\frac{bh}{2}}{2+4}=
    \frac{\frac{6*3}{2}}{6}=
    \frac{9}{6}=1.5$ \\

  \item{$f(x)=1-\sqrt{1-x^{2}}\,\,\,[-1,1]$} \\

    Test of Continuity:

    $\lim_{x\to -1}f(x)=f(-1)=
    \lim_{x\to 1}f(x)=f(1)=0$\\
    $\therefore f(x)$ is continuous on $[-1,1]$ (Fundamental Theorem of Calculus holds.) \\

    $f_{avg}=\frac{bh-\frac{\pi\,r^2}{2}}{1+1}=
    \frac{2*1-\frac{\pi}{2}}{2}=
    1-\frac{\pi}{4}\approx0.215$

\end{enumerate}
\end{document}

% Test of Differentiability:
% $\lim_{x \to 2}f(x)=f(2)=1,
%\lim_{x \to 5}f(x)=f(5)=4\\$
%$\therefore F(x)$ is continuous on $[2,5]$ and differentiable on $(2,5)$ (Fundamental Theorem of Calculus holds.) \\
