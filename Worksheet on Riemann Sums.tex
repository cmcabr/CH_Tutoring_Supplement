\documentclass[10pt, letterpaper]{report}
\usepackage[utf8]{inputenc}
\usepackage{mathtools , amssymb , amsthm}
\usepackage{textcomp, gensymb, graphicx, wrapfig, multicol}
\usepackage[letterpaper]{geometry}
  \geometry{top=1in, bottom=1in, left=1in, right=1in}
\graphicspath{ {/home/lowebang/Pictures/} }
\usepackage{fancyhdr}
  \pagestyle{fancy}
  \lhead{}
  \chead{}
  \rhead{Cabrera \thepage}
  \lfoot{}
  \cfoot{}
  \rfoot{\LaTeX}
  \renewcommand{\headrulewidth}{1pt}
  \renewcommand{\footrulewidth}{1pt}
  \setlength\headsep{0.333in}

\title{Calculus BC - Worksheet on Riemann Sums}
\author{Craig Cabrera}
\date{15 December 2021}

\begin{document}
\maketitle
Work the following on \textbf{\underline{notebook paper}}. Use your calculator, and give decimal answers correct to three decimal places. \\

On problems 1-2, estimate the area bounded by the curve and the $x$-axis on the given interval using the indicated numver of subintervals by finding:

\begin{enumerate}
  \item{$y=\sqrt{x}$, $[0,1]$, $n=4$ subintervals}
    \begin{enumerate}
      \item{a left Riemann sum} \\

        $\sum_{n=1}^{4}\frac{1}{4}\sqrt{\frac{n}{4}-\frac{1}{4}}=\frac{\sqrt{0}}{4}+\frac{\sqrt{0.25}}{4}+\frac{\sqrt{0.5}}{4}+\frac{\sqrt{0.75}}{4}=
        \frac{1+\sqrt{2}+\sqrt{3}}{8}\approx0.518$ \\

      \item{a right Riemann sum} \\

        $\sum_{n=1}^{4}\frac{1}{4}\sqrt{\frac{n}{4}}=
        \frac{\sqrt{0.25}}{4}+\frac{\sqrt{0.5}}{4}+\frac{\sqrt{0.75}}{4}+\frac{\sqrt{1}}{4}=
        \frac{1+\sqrt{2}+\sqrt{3}+2}{8}\approx0.768$ \\

      \item{a midpoint Riemann sum} \\

        $\sum_{n=1}^{4}\frac{1}{4}\sqrt{\frac{n}{4}+\frac{1}{8}}=
        \frac{1+\sqrt{6}+\sqrt{10}+\sqrt{14}}{16}\approx0.673$ \\

      \item{Additional Riemann Integration for Reference} \\

        $\int_{0}^{1}\sqrt{x}\,dx=
        [\frac{2}{3}x^{\frac{3}{2}}]_{0}^{1}=
        (\frac{2}{3}(1)^{\frac{3}{2}})-(\frac{2}{3}(0)^{\frac{3}{2}})=
        \frac{2}{3}=0.667$ \\

    \end{enumerate}
  \item{$y=\frac{1}{x}$, $[1,3]$, $n=4$ subintervals}
    \begin{enumerate}
      \item{a left Riemann sum} \\

        $\sum_{n=1}^{4}\frac{1}{n+1}=
        \frac{1}{2}+\frac{1}{3}+\frac{1}{4}+\frac{1}{5}=
        \frac{60+40+30+24}{120}=
        \frac{77}{60}\approx1.283$ \\

      \item{a right Riemann sum} \\

        $\sum_{n=1}^{4}\frac{1}{n+2}=
        \frac{1}{3}+\frac{1}{4}+\frac{1}{5}+\frac{1}{6}=
        \frac{40+30+24+20}{120}=
        \frac{57}{60}\approx0.950$ \\

      \item{a midpoint Riemann sum} \\

        $\sum_{n=1}^4\frac{2}{2n+3}=
        \frac{2}{5}+\frac{2}{7}+\frac{2}{9}+\frac{2}{11}=
        \frac{1386+990+770+630}{3465}=
        \frac{3776}{3465}\approx1.090$ \\

      \item{Additional Riemann Integration for Reference} \\

        $\int_{1}^{3}\frac{1}{x}\,dx=
        [\ln{x}]_{1}^3=
        \ln{3}-\ln{1}=\ln{3}=1.099$ \\

    \end{enumerate}
\pagebreak
  \item{Estimate the area bounded by the curve and the $x$-axis on $[1,6]$ using the 5 equal subintervals by finding:}
    \begin{enumerate}
      \item{a left Riemann sum} \\

        $\sum_{n=1}^{5}f(x)=
        2+3+4+3+2=14$ \\

      \item{a right Riemann sum} \\

        $\sum_{n=1}^{5}f(x)=
        3+4+3+2+1=13$ \\

      \item{a midpoint Riemann sum} \\

        $\sum_{n=1}^{5}f(x)\,\Delta x=
        1.5+3+4+3+2=13.5$ \\

      \item{Additional Riemann Integration for Reference} \\

        Estimated Polynomial: $f(x)=-0.05x^{5}+0.9167x^{4}-6.25x^{3}+19.083x^{2}-24.7x+13$ \\

        $\int_{1}^{6} f(x)\,dx\approx-0.05[\frac{x^{6}}{6}]_{1}^{6}+0.9167[\frac{x^5}{5}]_{1}^{6}-6.25[\frac{x^{4}}{4}]_{1}^{6}+19.083[\frac{x^{3}}{3}]_{1}^{6}-24.7[\frac{x^2}{2}]_{1}^{6}+13[x]_{1}^{6}=13.604$ \\

    \end{enumerate}
\pagebreak
  \item{Oil is leaking out of a tank. The rate of flow is measured every two hours for a 12-hour period, and the data is listed in the table below.
    \begin{center}
      \begin{tabular}{| c | c | c | c | c | c | c | c |}
        \hline
        Time (hr) & 0 & 2 & 4 & 6 & 8 & 10 & 12 \\
        \hline
        Rate (gal/hr) & 40 & 38 & 36 & 30 & 26 & 18 & 8 \\
        \hline
      \end{tabular}
    \end{center}}
    \begin{enumerate}
      \item{Draw a possibe graph for the data given in the table.} \\
        \includegraphics[scale=0.31]{Untitled.png}
      \item{Estimate the number of gallons of oil that have leaked out of the tank during the 12-hour period by finding a \underline{left} Riemann sum with \underline{three} equal subintervals.} \\

        $\sum_{n=0}^{3}4f(x)=
        4(40+36+26)=
        160+144+104=408$ gallons \\

      \item{Estimate the number of gallons of oil that have leaked out of the tank during the 12-hour period by finding a \underline{right} Riemann sum with \underline{three} equal subintervals.} \\

        $\sum_{n=0}^{3}4f(x)=
        4(36+26+8)=
        144+104+32=280$ gallons \\

      \item{Estimate the number of gallons of oil that have leaked out of the tank during the 12-hour period by finding a \underline{midpoint} Riemann sum with \underline{three} equal subintervals.} \\

        $\sum_{n=0}^{3}4f(x)=
        4(38+30+18)=
        152+120+72=344$ gallons \\

      \item{Additional Riemann Integration for Reference} \\

        Estimated Polynomial: $r(t)=-\frac{1}{5}t^{2}-0.32t+40$ \\

        $\int_{0}^{12}r(t)dt\approx
        -\frac{1}{5}[\frac{t^{3}}{3}]_{0}^{12}-0.32[\frac{t^{2}}{2}]_{0}^{12}+40[t]_{0}^12=
        -\frac{1}{5}[576]-0.32[72]+40[12]=341.76$ gallons \\
    \end{enumerate}
\pagebreak
  \item{Oil is being pumped into a tank over a 12-hour period. The tank contains 120 gallons of oil when $t=0$. The rate at which oil is flowing into the tank at various times is modeled by a differentiable function $R$ for $0\leq t\leq 12$, where $t$ is measured in hours and $R(t)$ is measured in gallons per hours. Values of $R(t)$ at selected values of time $t$ are shown in the table below.
    \begin{center}
      \begin{tabular}{| c | c | c | c | c | c |}
        \hline
        $t$ (hours) & 0 & 3 & 5 & 9 & 12 \\
        \hline
        $R(t)$ (gallons per hour) & 8.9 & 6.8 & 6.4 & 5.9 & 5.7 \\
        \hline
      \end{tabular}
    \end{center}}
    \begin{enumerate}
      \item{Estimate the number of gallons of oil in the tank at $t=12$ hours by using a left Riemann sum with four subintervals and values from the table. Show the computations that led to your answer.} \\

        $\sum_{n=0}^{4}R(t)\Delta t=
        3(8.9)+2(6.8)+4(6.4)+3(5.9)=
        26.7+13.6+25.6+17.7=83.6$ gallons \\

        $120+83.6=203.6$ gallons \\

      \item{Estimate the number of gallons of oil in the tank at $t=12$ hours by using a right Riemann sum with four subintervals and values from the table. Show the computations that led to your answer.} \\

        $\sum_{n=0}^{4}R(t)\Delta t=
        3(6.8)+2(6.4)+4(5.9)+3(5.7)=
        20.4+12.8+23.6+17.1=73.9$ gallons \\

        $120+73.9=193.9$ gallons \\
      \item{Addition Riemann Integration for Reference} \\

        Estimated Polynomial: $R(t)=0.034t^{2}-0.65t+8.9$ \\

        $\int_{0}^{12}R(t)dt\approx
        0.034[\frac{t^{3}}{3}]_{0}^{12}-0.65[\frac{x^{2}}{2}]_{0}^{12}+8.9[x]_{0}^{12}=
        0.034[576]-0.65[72]+8.9[12]=79.584$ gallons \\

        $120+79.584=199.584$ gallons \\

    \end{enumerate}
\pagebreak
  \item{A hot cup of coffee is taken into a classroom and set on a desk to cool. When $t=0$, the temperature of the coffee is 113\degree F. The rate at which the temperature of the coffee is dropping is modeled by a differentiable function $R$ for $0\leq t\leq 8$, where $R(t)$ is measured in degrees Fahreheit per minute and $t$ is measured in minutes. Values of $R(t)$ at selected values of time $t$ are shown in the table below.
    \begin{center}
      \begin{tabular}{| c | c | c | c | c |}
        \hline
        $t$ (minutes) & 0 & 3 & 5 & 8 \\
        \hline
        $R(t)$ (\degree F/min.) & 5.5 & 2.7 & 1.6 & 0.8 \\
        \hline
      \end{tabular}
    \end{center}}
    \begin{enumerate}
      \item{Estimate the temperature of the coffee at $t=8$ hours by using a left Riemann sum with three subintervals and values from the table. Show the computations that led to your answer.} \\

        $\sum_{n=0}^{4}R(t)\Delta\,t=
        3(5.5)+2(2.7)+3(1.6)=
        16.5+5.4+4.8=26.7$\degree F \\

        $113-26.7=86.3$ \degree F \\

      \item{Estimate the temperature of the coffee at $t=8$ hours by using a left Riemann sum with three subintervals and values from the table. Show the computations that led to your answer.} \\

        $\sum_{n=0}^{4}R(t)\Delta\,t=
        3(2.7)+2(1.6)+3(0.8)=
        8.1+3.2+2.4=13.7$\degree F \\

        $113-13.7=99.3$ \degree F \\

      \item{Additional Riemann Integration for Reference} \\

        Estimated Polynomial: $R(t)=-0.0025x^{3}+0.097x^{2}-1.201x+5.5$ \\

        $\int_{0}^{8}R(t)dt\approx
        -0.0025[\frac{x^{4}}{4}]_{0}^{8}+0.097[\frac{x^{3}}{3}]_{0}^{8}-1.201[\frac{x^{2}}{2}]_{0}^{8}+5.5[x]_{0}^{8}=
        -2.56+16.555-38.432+44=19.563$\degree F \\

        $113-19.563=93.437$ \degree F \\
\hline
    \end{enumerate}
  \item{(Modification of 2004 Form B AB 3/ BC 3) \\
  A test plane flies in a straight line with positive velocity $v(t)$, in miles per minute at time $t$ minutes, where $v$ is a differentiable function of $t$. Selected values of $v(t)$ for $0\leq t\leq 40$ are shown in the table below.
    \begin{center}
      \begin{tabular}{| c | c | c | c | c | c | c | c | c | c |}
        \hline
        $t$ (min) & 0 & 5 & 10 & 15 & 20 & 25 & 30 & 35 & 40 \\
        \hline
        $v(t)$ (mpm) & 7.0 & 9.2 & 9.5 & 7.0 & 4.5 & 2.4 & 2.4 & 4.3 & 7.2 \\
        \hline
      \end{tabular}
    \end{center}}

    Use a midpoint Riemann sum with four subintervals of equal length and values from the table to approximate the distance traveled by the plane during the 40 minutes. Show the computations that lead to your answer. \\

    $\sum_{n=0}^{4}10v(t)=
    10(9.2)+10(7.0)+10(2.4)+10(4.3)=
    92+70+24+43=229$ miles \\

    Did not use estimate integration because:
    \begin{itemize}
      \item{AP Reference is accessible for comparison.}
      \item{Computationally Estimated Equation contained coefficients times $10^{-5}$ or smaller power.}
    \end{itemize}
\end{enumerate}
\end{document}
