\documentclass[10pt, letterpaper]{report}
\usepackage[letterpaper]{geometry}
  \geometry{top=1in, bottom=1in, left=1in, right=1in}
\usepackage[utf8]{inputenc}
\usepackage{textcomp, gensymb, mathtools , amssymb , amsthm, graphicx}
  \graphicspath{ {/home/lowebang/Pictures/} }
\usepackage{fancyhdr}
  \pagestyle{fancy}
  \lhead{}
  \chead{}
  \rhead{Cabrera \thepage}
  \lfoot{}
  \cfoot{}
  \rfoot{\LaTeX}
  \renewcommand{\headrulewidth}{1pt}
  \renewcommand{\footrulewidth}{1pt}
  \setlength\headsep{0.333in}

\title{Calculus BC - Worksheet 2 on Functions Defined by Integrals}
\author{Craig Cabrera}
\date{23 January 2022}

\begin{document}
\maketitle
Work the following on \textbf{\underline{notebook paper}}.
\begin{enumerate}
  \item{Find the equation of the tangent line to the curve $y=F(x)$ where $F(x)=\int_{1}^{x}{\sqrt[3]{t^{2}+7}}\,dt$ at the point on the curve where $x=1$.} \\

    $y - F(1) = F'(1)(x - 1)$ \\

    $y - \int_{1}^{1}{\sqrt[3]{t^{2}+7}}\,dt = \sqrt[3]{t^{2}+7}(x - 1)$ \\

    $y = \sqrt[3]{8}(x - 1)$ \\

    $y = 2x - 2$

  \item{Suppose that $5x^{3}+40=\int_{c}^{x}{f(t)}\,dt$.}
  \begin{enumerate}
    \item{What is $f(x)$?} \\

      $f(x)=\frac{d}{dx}\int_{c}^{x}{f(t)}\,dt=\frac{d}{dx}(5x^{3}+40)=15x^{2}$ \\

    \item{Find the value of $c$.} \\

      $\int_{c}^{x}{f(t)}\,dt=[5x^{3}]_{c}^{x}=5x^{3} - 5c^{3} = 5x^{3} + 40$ \\

      $5c^{3}=-40$ \\

      $c^{3}=-8$ \\

      $c = -2$ \\

  \end{enumerate}
  \item{If $F(x)=\int_{-4}^{x}{(t-1)^{2}(t+3)}\,dt$, for what values of $x$ is $F$ decreasing? Justify your answer.} \\

    To determine where our initial function is decreasing, we must find where our derivative is less than zero, indicating a negative slope. \\

    $\frac{d}{dx}\int_{-4}^{x}{(t-1)^{2}(t+3)}\,dt = (x-1)^{2}(x+3)$ \\

    $x-1<0\rightarrow x<1$ \\

    $x+3<0\rightarrow x<-3$ \\

    $\therefore$ Because $F'(x)<0$ at $x<-3$, $F(x)$ is decreasing at $x<-3$. \\
\pagebreak
  \item{The function $F$ is defined for all $x$ by $F(x)=\int_{0}^{x^{2}}{\sqrt{t^{2}+8}}\,dt$.}
  \begin{enumerate}
    \item{Find $F'(x)$.} \\

      $F'(x)=\frac{d}{dx}\int_{0}^{x^{2}}{\sqrt{t^{2}+8}}\,dt=2x\sqrt{x^{4}+8}$ \\

    \item{Find $F'(1)$.} \\

      $F'(1)=2\sqrt{1^{4}+8}=2\sqrt{9}=6$ \\

    \item{Find $F''(x)$.} \\

      $F''(x)=\frac{d}{dx}2x\sqrt{x^{4}+8}=2\sqrt{x^{4}+8}+2x\left(\frac{2x^{3}}{\sqrt{x^{4}+8}}\right)
      =2\sqrt{x^{4}+8}+\frac{4x^{4}}{\sqrt{x^{4}+8}}$ \\

    \item{Find $F''(1)$.} \\

      $F''(1)=2\sqrt{1^{4}+8}+\frac{4(1)^{4}}{\sqrt{1^{4}+8}}=2\sqrt{9}+\frac{4}{\sqrt{9}}=
      6+\frac{4}{3}=7\frac{1}{3}=\frac{22}{3}$ \\

  \end{enumerate}
\pagebreak
  \item{The function $F$ is defined for all $x$ by $F(x)=\int_{0}^{x}{f(t)}\,dt$, where $f$ is the function graphed in the figure. The graph of $f$ is made up of straight lines and a semicircle.}
  \begin{enumerate}
    \item{For what values of $x$ is $F$ decreasing? Justify your answer. } \\

      Because the graph of the derivative $f$ is negative at $(-5,-3.5)\cup(2,5)$, we can determine that $F$ is decreasing on $(-5,-3.5)\cup(2,5)$ \\

    \item{For what values of $x$ does $F$ have a local maximum? A local minimum? Justify your answer. } \\

      Because $f$ changes from positive to negative at $x=2$, we can determine that $F$ has a local maximum at $x=2$. Because $f$ changes from negative to positive at $x=-3.5$, we can determine that $F$ has a local minimum at $x=-3.5$ \\

    \item{Evaluate $F(2)$, $F'(2)$, and $F''(2)$.} \\

      $F(2)=\int_{0}^{2}{f(t)}\,dt=\int_{0}^{1}{f(t)}\,dt+\int_{1}^{2}{f(t)}\,dt=
      \frac{b_{1}+b_{2}}{2}h + \frac{bh}{2} = \frac{2+3}{2} + \frac{3*1}{2} = \frac{5+3}{2} = 4$ \\

      $F'(x)=\frac{d}{dx}\int_{0}^{x}{f(t)}\,dt = f(x)\rightarrow F'(2) = f(2) = 0$ \\

      $F''(x) = \frac{d}{dx}f(x) = \frac{y_{2}-y_{1}}{x_{2}-x_{1}}=\frac{-3-3}{3-1}=-\frac{6}{2}=-3$ \\

    \item{Write an equation of the line tangent to the graph of $F$ at $x=4$.} \\

      $y - F(4) = f(4)(x - 4)$ \\

      $y - \int_{0}^{4}{f(t)}\,dt = f(4)(x - 4)$ \\

      $y = -2x + 8$ \\

    \item{For what values of $x$ does $F$ have an inflection point? Justify your answer.} \\

      Because the graph of $f$ changes from increasing to decreasing or decreasing to increasing at $x=-3, x=-2, x=1$ and , $x=3$, we can determine that $F$ has inflection points at $x=-3, x=-2, x=1$ and , $x=3$.\\

  \end{enumerate}
\pagebreak
  \item{The graph of a function $f$ consists of a semicircle and two line segments as shown on the right. Let $g(x)=\int_{1}^{x}{f(t)}\,dt$.}
  \begin{enumerate}
    \item{Find $g(1)$, $g(3)$, $g(-1)$.} \\

      $g(1)=\int_{1}^{1}{f(t)}\,dt=0$ \\

      $g(3)=\int_{1}^{3}{f(t)}\,dt=\frac{bh}{2}=\frac{-1*2}{2}=-1$ \\

      $g(-1)=\int_{1}^{-1}{f(t)}\,dt=-\int_{-1}^{1}{f(t)}\,dt=-\frac{\pi\,r^{2}}{4}=
      -\frac{4\pi}{4}=-\pi$ \\

    \item{On what interval(s) of $x$ is $g$ decreasing? Justify your answer.} \\

      Because the graph of the derivative $f$ is negative on $(1,3)$, we can determine that $g$ is decreasing on $(1,3)$. \\

    \item{Find all values of $x$ on the open interval $(-3, 4)$ at which $g$ has a relative minimum. Justify your answer.} \\

      Because $f$ changes from negative to positive at $x=3$, we can determine that $g$ has a local minimum at $x=3$ on the interval $(-3,4)$. \\

    \item{Find the absolute maximum value of $g$ on the interval $[-3,4]$ at the value of $x$ at which it occurs. Justify your answer.} \\

      Let us conduct a Candidate Test using our critical points (the values where $f(x)=0$) by graphing our $x$-values and their respective values on $g(x)$.

      \begin{center}
        \begin{tabular}{| c | c | c | c |}
          \hline
          $x$ & -3 & 1 & 3 \\
          \hline
          $g(x)$ & $-2\pi$ & $0$ & $-1$ \\
          \hline
        \end{tabular}
      \end{center}

      The absolute maximum value of $g$ on the closed interval $[-3,4]$ is $0$ at $x=1$ by the Candidate Test.\\

    \item{On what interval(s) of $x$ is $g$ concave up? Justify your answer.} \\

      To determine where a function is concave up, we must find on the graph where $f$ is increasing, denoting where $f'$ is positive. Because $f$ is increasing on $(-3,-1)\cup(2,4)$, we can determine that $g$ is concave up on $(-3,-1)\cup(2,4)$. \\

    \item{For what value(s) of $x$ does the graph of $g$ have an inflection point? Justify your answer.} \\

      Because the graph of $f$ changes from increasing to decreasing or decreasing to increasing at $x=-1$ and $x=2$, we can determine that $g$ has inflection points at $x=-1$ and $x=2$.\\

    \item{Write an equation for the line tangent to the graph of $g$ at $x=-1$.} \\

      $y - g(-1) = f(-1)(x + 1)$ \\

      $y + \pi = 2x + 2$ \\

      $y = 2x + 2 - \pi$ \\

  \end{enumerate}
\pagebreak
  \item{The graph of the velocity $v(t)$, in ft/sec, of a car traveling on a straight road, for $0\leq t\leq 35$, is shown in the figure.}
  \begin{enumerate}
    \item{Find the average acceleration of the car, in ft / sec$^{2}$, over the interval $0\leq t\leq 35$.} \\

      $a(t) = [\frac{dv}{dt}]_{0}^{35}$ \\

      $a(35) = \frac{v_{2}-v_{1}}{t_{2}-t_{1}} = \frac{30}{35} = \frac{6}{7}$ ft/sec$^{2}$ \\

    \item{Find an approximation for the acceleration of the car, in ft / sec$^{2}$, at $t=20$. Show your computations.} \\

      $a(20) = \frac{v_{2}-v_{1}}{t_{2}-t_{1}} = \frac{30 - 40}{25 - 20} = -\frac{10}{5} = -2$ ft / sec$^{2}$ \\

    \item{Approximate $\int_{5}^{35}{v(t)}\,dt$ with a Riemann sum, using the midpoints of three subintervals of equal length. Explain the meaning of this integral.} \\

      $\int_{5}^{35}{v(t)}\,dt \approx \sum_{i=1}^{3}{10v(t)} = 10(30+40+20) = 900$ ft / sec$^{2}$ \\

      The integral represents the approximate distance the car has traveled between 5 and 35 seconds.

  \end{enumerate}
\end{enumerate}
\end{document}
