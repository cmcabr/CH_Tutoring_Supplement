\documentclass[10pt, letterpaper]{report}
\usepackage[letterpaper]{geometry}
  \geometry{top=1in, bottom=1in, left=1in, right=1in}
\usepackage[utf8]{inputenc}
\usepackage{textcomp, gensymb, mathtools , amssymb , amsthm, graphicx}
  \graphicspath{ {/home/lowebang/Pictures/} }
\usepackage{fancyhdr}
  \pagestyle{fancy}
  \lhead{}
  \chead{}
  \rhead{Cabrera \thepage}
  \lfoot{}
  \cfoot{}
  \rfoot{\LaTeX}
  \renewcommand{\headrulewidth}{1pt}
  \renewcommand{\footrulewidth}{1pt}
  \setlength\headsep{0.333in}

\title{Calculus BC - Worksheet on Second Fundamental Theorem \\ \& \\ Functions Defined by Integrals}
\author{Craig Cabrera}
\date{20 January 2022}

\begin{document}
\maketitle
\begin{enumerate}
  \item{Evaluate.}
  \begin{enumerate}
    \item{$\frac{d}{dx}\int_{3}^{x}{\frac{\sin{t}}{t}}\,dt$} \\

      $\frac{d}{dx}\int_{3}^{x}{\frac{\sin{t}}{t}}\,dt=
      \frac{\sin{x}}{x}$ \\

    \item{$\frac{d}{dx}\int_{\pi}^{x}{e^{-t^{2}}}\,dt$} \\

      $\frac{d}{dx}\int_{\pi}^{x}{e^{-t^{2}}}\,dt=
      e^{-x^{2}}$ \\

    \item{$\frac{d}{dx}\int_{1}^{\cos{x}}{\frac{1}{t}}\,dt$} \\

      $\frac{d}{dx}\int_{1}^{\cos{x}}{\frac{1}{t}}\,dt=
      \left( \frac{1}{\cos{x}}\right)\left( -\sin{x}\right)=
      -\tan{x}$ \\

    \item{$\frac{d}{dx}\int_{x}^{2}{\ln{(t^{2}})}\,dt$} \\

      $\frac{d}{dx}\int_{x}^{2}{\ln{(t^{2}})}\,dt =
      -\frac{d}{dx}\int_{2}^{x}{\ln{(t^{2}})}\,dt =
      -2\ln{x}$ \\

    \item{$\frac{d}{dx}\int_{-5}^{x^{2}}{\cos{(t^{3}})}\,dt$} \\

      $\frac{d}{dx}\int_{-5}^{x^{2}}{\cos{(t^{3}})}\,dt=
      \left( \cos{x^{6} }\right) \left( 2x \right)=
      2x\cos{x^{6}}$ \\

    \item{$\frac{d}{dx}\int_{\tan{x}}^{17}{\sin{(t^{4})}}\,dt$} \\

      $\frac{d}{dx}\int_{\tan{x}}^{17}{\sin{(t^{4})}}\,dt=
      -\frac{d}{dx}\int_{17}^{\tan{x}}{\sin{(t^{4})}}\,dt=
      \left( -\sin{(\tan^{4}{(x)})} \right)\left( \sec^{2}{x} \right)=
      -\sec^{2}{x}\sin{(tan^{4}{x})}$ \\

  \end{enumerate}
  \pagebreak
  \item{The graph of a function $f$ consists of a semicircle and two line segments as shown. Let $g$ be the function given by $g(x)=\int_{0}^{x}{f(t)}\,dt$}.
  \begin{enumerate}
    \item{Find $g(0)$, $g(3)$, $g(-2)$, and $g(5)$.} \\

      $g(0)=\int_{0}^{0}{f(t)}\,dt=0$ \\

      $g(3)=\int_{0}^{3}{f(t)}\,dt=
      \int_{0}^{2}{f(t)}\,dt + \int_{2}^{3}{f(t)}\,dt= \frac{\pi\,r^{2}}{4} + \frac{bh}{2} =
      \frac{\pi\,(2)^{2}}{4} - \frac{1*1}{2} =
      \pi - \frac{1}{2}$ \\

      $g(-2)=\int_{0}^{-2}{f(t)}\,dt=
      -\int_{-2}^{0}{f(t)}\,dt=
      -\int_{0}^{2}{f(t)}\,dt=-\pi$ \\

      $g(5)=\int_{0}^{5}{f(t)}\,dt=
      \int_{0}^{3}{f(t)}\,dt + \int_{3}^{4}{f(t)}\,dt + \int_{4}^{5}{f(t)}\,dt=
      g(3) + \frac{bh}{2} + \frac{bh}{2} =
      \left( \pi - \frac{1}{2} \right) - \frac{1*1}{2} + \frac{1*1}{2}= $ \\

      $\pi - \frac{1}{2} + \frac{1}{2} - \frac{1}{2} = \pi - \frac{1}{2}$ \\

    \item{Find all values of $x$ on the open interval $(-2,5)$ at which $g$ has a relative maximum. Justify your answers.} \\

      $g$ has a relative maximum at $x=2$ because $f$ changes from positive to negative, indicating a change on $g$ from increasing to decreasing. \\

    \item{Find the absolute minimum value of $g$ on the closed interval $[-2,5]$ and the value of $x$ at which it occurs. Justify your answer.} \\

      Let us conduct a Candidate Test using our critical points (the values where $f(x)=0$) by graphing our $x$-values and their respective values on $g(x)$.

      \begin{center}
        \begin{tabular}{| c | c | c | c |}
          \hline
          $x$ & -2 & 2 & 4 \\
          \hline
          $g(x)$ & $-\pi$ & $\pi$ & $\pi - 1$ \\
          \hline
        \end{tabular}
      \end{center}

      The absolute minimum value of $g$ on the closed interval $[-2,5]$ is $-\pi$ at $x=-2$ by the Candidate Test.\\

    \item{Write an equation for the line tangent to the graph of $g$ at $x=3$.} \\

      $y - g(x_{0}) = f(x_{0})(x - x_{0}) \rightarrow y - g(3) = f(3)(x - 3) $ \\

      $y - (\pi - \frac{1}{2}) = -(x - 3) \rightarrow y = -x + \frac{5}{2} + \pi$ \\

    \item{Find the $x$-coordinate of each point of inflection of the graph of $g$ on the open interval $(-2,5)$. Justify your answer.} \\

      The function $g$ has an inflection point at $x=0$ because $f$ changes from increasing to decreasing at this value. \\

      The function also has an inflection point at $x=3$ because $f$ changes from dereasing to increasing on this value. \\

    \item{Find the range of $g$.} \\

      We currently know the lower bound of the range of $g$ because we have solved for an absolute minimum. Our previous Candidate Test also shows an upper bound of the range of $g$, and using this table we can conclude the range of $g$ is $[-\pi, \pi]$.
  \pagebreak
  \end{enumerate}
  \item{Let $g(x)=\int_{0}^{x}{f(t)}\,dt$, where $f$ is the function whose graph is shown.}
  \begin{enumerate}
    \item{Evaluate $g(0)$, $g(1)$, $g(2)$, and $g(7)$.} \\

      $g(0) = \int_{0}^{0}{f(t)}\,dt = 0$ \\

      $g(1) = \int_{0}^{1}{f(t)}\,dt = bh = 2*1 = 2$ \\

      $g(2) = \int_{0}^{2}{f(t)}\,dt =
      \int_{0}^{1}{f(t)}\,dt + \int_{1}^{2}{f(t)}\,dt =
      g(1) + \frac{b_{1} + b_{2}}{2}h = 2 + \frac{2 + 4}{2}*1 = 2 + 3 = 5$ \\

      $g(7) = \int_{0}^{7}{f(t)}\,dt =
      \int_{0}^{2}{f(t)}\,dt + \int_{2}^{3}{f(t)}\,dt + \int_{3}^{5}{f(t)}\,dt + \int_{5}^{6}{f(t)}\,dt + \int_{6}^{7}{f(t)}\,dt = $ \\

      $g(2) + \frac{bh}{2} + \frac{bh}{2} + bh + \frac{bh}{2} = 5 + \frac{4*1}{2} - \frac{2*2}{2} - (2*1) - \frac{2*1}{2} =
      5 + 2 - 2 - 2 - 1 = 2$ \\

    \item{Write an equation for the line tangent to the graph of $g$ at $x=4$.} \\

      $y - g(x_{0}) = f(x_{0})(x - x_{0}) \rightarrow
      y - g(4) = f(4)(x - 4)$ \\

      $y - 6 = -(x-4) \rightarrow y = -x + 10$ \\

    \item{On what intervals is $g$ increasing? Decreasing? Justify your answer.} \\

      $g$ is increasing on the interval $(0,3)$ because all values of $g' = f$ are positive within this interval, and $g$ is decreasing on the interval $(3,7)$ because all values of $g' = f$ are negative within this interval. \\

    \item{Find all values of $x$ on the open interval $0<x<7$ at which $g$ has a relative maximum. Justify your answer.} \\

      $g$ has a relative maximum at $x=3$ because $f$ changes from positive to negative, indicating a change on $g$ from increasing to decreasing. \\

    \item{Where does $g$ have its absolute maximum value? What is the maximum value? Justify your answer.} \\

      For this question as well as the following one, let us conduct a Candidate Test using our critical points (the values where $f(x)=0$) by tabulating our $x$-values and their respective values on $g(x)$.

      \begin{center}
        \begin{tabular}{| c | c | c |}
          \hline
          $x$ & 3 & 7 \\
          \hline
          $g(x)$ & $7$ & $2$  \\
          \hline
        \end{tabular}
      \end{center}

      From this table, we can determine that the absolute maximum value of $g$ is 7 at $x=3$ by the Candidate Test.

    \item{Where does $g$ have its absolute minimum value? What is the minimum value? Justify your answer. } \\

      The absolute minimum value of $g$ is 2 at $x=7$ by the Candidate Test.

  \end{enumerate}
  \pagebreak
  \item{Let $g(x)=\int_{0}^{x}{f(t)}\,dt$, where $f$ is the function whose graph is shown.}
  \begin{enumerate}
    \item{On what intervals is $g$ decreasing? Justify.} \\

      $g$ is decreasing on the intervals $(1,2.5)\cup(4,\infty)$ because all values of $f$ are negative within this interval. \\

    \item{For what value(s) of $x$ does $g$ have a relative maximum? Justify.} \\

      $g$ has a relative maximum at $x=0.5$ and $x=3.25$ because $f$ changes from positive to negative, indicating a change on $g$ from increasing to decreasing. \\

    \item{On what intervals is $g$ concave down? Justify.} \\

      $g$ is concave down on the intervals $(0.5,1.75)\cup(3.25,\infty)$ because $f$ is decreasing on these intervals, indication downwards concavity on $g$. \\

    \item{At what values of $x$ does $g$ have an inflection point? Justify.} \\

      $g$ has inflection points at $x=0.5$, $x=1.75$, and $x=3.25$ because $f$ changes from increasing to decreasing or decreasing to increasing, indicating a value of $f' = 0$ at these points. \\

  \end{enumerate}
  \pagebreak
  \item{The graph of the function $f$, consisting of three line segments, is shown on the second page. Let $g(x)=\int_{1}^{x}{f(t)}\,dt$,}
  \begin{enumerate}
    \item{Find $g(2)$, $g(4)$, and $g(-2)$.} \\

      $g(2) = \int_{1}^{2}{f(t)}\,dt = \frac{b_{1}+b_{2}}{2}h =
      \frac{3+1}{2}*1 = 2$ \\

      $g(4) = \int_{1}^{2}{f(t)}\,dt + \int_{2}^{3}{f(t)}\,dt + \int_{3}^{4}{f(t)}\,dt =
      g(2) + \frac{bh}{2} + \frac{bh}{2} =
      2 + \frac{1*1}{2} - \frac{1*1}{2} = 2$ \\

      $g(-2) = \int_{1}^{-2}{f(t)}\,dt =
      -\int_{-2}^{1}{f(t)}\,dt = -\frac{bh}{2} =
      -\frac{3*3}{2} = -\frac{9}{2}$ \\

    \item{Find $g'(0)$ and $g'(3)$.} \\

      $g'(x) = \frac{d}{dx}\int_{1}^{x}{f(t)}\,dt = f(x)$ \\

      $g'(0) = f(0) = 2$ \\

      $g'(3) = f(3) = 0$ \\

    \item{Find the instantaneous rate of change of $g$ with respect to $x$ at $x=2$.} \\

      $g'(2) = f(2) = 1$ \\

    \item{Find the absolute maximum value of $g$ on the interval $[-2,4]$. Justify.} \\

      For this question, let us conduct a Candidate Test using our critical points (the values where $f(x)=0$) by tabulating our $x$-values and their respective values on $g(x)$.

      \begin{center}
        \begin{tabular}{| c | c | c |}
          \hline
          $x$ & -2 & 3 \\
          \hline
          $g(x)$ & $-\frac{9}{2}$ & $\frac{5}{2}$  \\
          \hline
        \end{tabular}
      \end{center}

      Based upon these values, the absolute maximum value of $g$ on the interval $[-2,4]$ is $\frac{5}{2}$ at $x=3$. \\

    \item{The second derivative of $g$ is not defined at $x=1$ and $x=2$. Which of these values are $x$-coordinates of points of inflection of the graph of $g$? Justify.} \\

      Of the two $x$-values, $x=1$ contains a point of inflection of the graph of $g$, because the function $f$ changes from increasing to decreasing, indicating a change in $f'$ from positive to negative, or a moment where $x=0$. $x=2$ is not a point of inflection as it continually decreases, indicating negative values on the function $f'$. \\
  \end{enumerate}
\end{enumerate}
\end{document}
