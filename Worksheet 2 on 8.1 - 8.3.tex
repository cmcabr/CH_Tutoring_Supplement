\documentclass[10pt, letterpaper]{report}
\usepackage[letterpaper]{geometry}
  \geometry{top=1in, bottom=1in, left=1in, right=1in}
\usepackage[utf8]{inputenc}
\usepackage{textcomp, gensymb, mathtools , amssymb , amsthm, graphicx}
  \graphicspath{ {/home/lowebang/Pictures/} }
\usepackage{fancyhdr}
  \pagestyle{fancy}
  \lhead{}
  \chead{}
  \rhead{Cabrera \thepage}
  \lfoot{}
  \cfoot{}
  \rfoot{\LaTeX}
  \renewcommand{\headrulewidth}{1pt}
  \renewcommand{\footrulewidth}{1pt}
  \setlength\headsep{0.333in}

\title{Calculus BC - Worksheet 2 on 8.1 -- 8.3}
\author{Craig Cabrera}
\date{7 February 2022}

\begin{document}
\maketitle
\begin{center}
  \textbf{\underline{Relevant Formulas:}} \\
  \includegraphics[scale=0.5]{Triangle.png}
  \includegraphics[scale=0.4]{trigo_functions.png} \\
\end{center}
\hline

$$ \sin^{2}{\theta}+\cos^{2}{\theta}=x^{2}+y^{2}=1$$ \\
$$ \frac{\sin^{2}{\theta}+\cos^{2}{\theta}}{\cos^{2}{\theta}}=\frac{1}{\cos^{2}{\theta}}\rightarrow
   \tan^{2}{\theta}+1=\sec^{2}{\theta}$$ \\
$$ \frac{\sin^{2}{\theta}+\cos^{2}{\theta}}{\sin^{2}{\theta}}=\frac{1}{\sin^{2}{\theta}}\rightarrow
   1+\cot^{2}{\theta}=\csc^{2}{\theta}$$ \\
\hline
$$ \cos{\left(\alpha\pm\beta\right)}=\cos{\alpha}\cos{\beta}\mp\sin{\alpha}\sin{\beta}\rightarrow
   \cos{\left(2\theta\right)}=\cos^{2}{\theta}-\sin^{2}{\theta}=
   \left(1-\sin^{2}{\theta}\right)-\sin^{2}{\theta}=
   \cos^{2}{\theta}-\left(1-\cos^{2}{\theta}\right)$$ \\

$$ \frac{1}{2}\left(\cos{\left(\alpha-\beta\right)}+\cos{\left(\alpha+\beta\right)}\right)=
   \frac{1}{2}\left(\left(\cos{\alpha}\cos{\beta}+\sin{\alpha}\sin{\beta}\right) +
   \left(\cos{\alpha}\cos{\beta}-\sin{\alpha}\sin{\beta}\right)\right)=
   \frac{2}{2}\left(\cos{\alpha}\cos{\beta}\right)$$ \\
$$ \frac{1}{2}\left(\cos{\left(\alpha-\beta\right)}-\cos{\left(\alpha+\beta\right)}\right)=
   \frac{1}{2}\left(\left(\cos{\alpha}\cos{\beta}+\sin{\alpha}\sin{\beta}\right) -
   \left(\cos{\alpha}\cos{\beta}-\sin{\alpha}\sin{\beta}\right)\right)=
   \frac{2}{2}\left(\sin{\alpha}\sin{\beta}\right)$$ \\
$$ \cos{\left(2\theta\right)}=\left(1-\sin^{2}{\theta}\right)-\sin^{2}{\theta}=
   1-2\sin^{2}{\theta}\therefore \sin^{2}{\theta}=\frac{1-\cos{\left(2x\right)}}{2}$$ \\
$$ \cos{\left(2\theta\right)}=\cos^{2}{\theta}-\left(1-\cos^{2}{\theta}\right)=
   2\cos^{2}{\theta}-1\therefore \cos^{2}{\theta}=\frac{1+\cos{\left(2x\right)}}{2}$$
\hline

$$ \sin{\left(\alpha\pm\beta\right)}=\sin{\alpha}\cos{\beta}\pm\cos{\alpha}\sin{\beta}$$ \\
$$ \frac{1}{2}\left(\sin{\left(\alpha-\beta\right)}+\sin{\left(\alpha+\beta\right)}\right)=
   \frac{1}{2}\left(\left(\sin{\alpha}\cos{\beta}-\cos{\alpha}\sin{\beta}\right)+
   \left(\sin{\alpha}\cos{\beta}+\cos{\alpha}\sin{\beta}\right)\right)=
   \frac{2}{2}\left(\sin{\alpha}\cos{\beta}\right)$$


\pagebreak
Work the following on \textbf{\underline{notebook paper}}. \textbf{\underline{No calculator.}}
\begin{enumerate}
  \item{$\int{\sec^{6}{(4x)}\tan{(4x)}}\,dx$} \\
  
    Let $\theta=4x\therefore d\theta=4dx$ \\
    
    $\int{\sec^{6}{(4x)}\tan{(4x)}}\,dx=
    \frac{1}{4}\int{\sec^{4}{\theta}\tan{\theta}}\,d\theta$ \\
    
    Let $u=\sec{\theta}\therefore du=\sec{\theta}\tan{\theta}d\theta$ \\
    
    $\frac{1}{4}\int{\sec^{6}{\theta}\tan{\theta}}\,d\theta=
    \frac{1}{4}\int{u^{5}}\,du=
    \frac{1}{24}u^{6}+C=
    \frac{1}{24}\sec^{6}{(4x)}+C$ \\
    
  \item{$\int{\tan^{5}{(3x)}\sec^{2}{(3x)}}\,dx$} \\
  
    Let $\theta=3x\therefore d\theta=3dx$ \\
    
    $\int{\tan^{5}{(3x)}\sec^{2}{(3x)}}\,dx=
    \frac{1}{3}\int{\tan^{5}{\theta}\sec^{2}{\theta}}\,d\theta$ \\
    
    Let $u=\tan{\theta}\therefore du=\sec^{2}{\theta}$ \\
    
    $\frac{1}{3}\int{\tan^{5}{\theta}\sec^{2}{\theta}}\,d\theta=
    \frac{1}{3}\int{u^{5}}\,du=
    \frac{1}{18}u^{6}+C=
    \frac{1}{18}\tan^{6}{(3x)}+C$ \\
    
  \item{$\int{\cos^{3}{(2x)}\sin^{2}{(2x)}}\,dx$} \\
  
    $\int{\cos^{3}{(2x)}\sin^{2}{(2x)}}\,dx=
    \int{\cos^{2}{(2x)}\sin^{2}{(2x)}\cos{(2x)}}\,dx=
    \int{(1-\sin^{2}{(2x)})\sin^{2}{(2x)}\cos{(2x)}}\,dx$ \\
    
    Let $u=\sin{(2x)}\therefore du=2\cos{(2x)}dx$ \\
    
    $\int{(1-\sin^{2}{(2x)})\sin^{2}{(2x)}\cos{(2x)}}\,dx=
    \frac{1}{2}\int{(1-u^{2})u^{2}}\,du=
    \frac{1}{2}\left(\int{u^{2}}\,du-\int{u^{4}}\,du\right)=
    \frac{1}{6}u^{3}-\frac{1}{10}u^{5}+C=$ \\
    
    $\frac{1}{6}\sin^{3}{(2x)}-\frac{1}{10}\sin^{5}{(2x)}+C$ \\
    
  \item{$\int{\frac{2x-3}{x^{2}-10x+41}}\,dx$} \\
  
    Let $u=x^{2}-10x+41\therefore du=(2x-10)dx$ \\
    
    $\int{\frac{2x-3}{x^{2}-10x+41}}\,dx=
    \int{\frac{du}{u}}-7\int{\frac{dx}{x^{2}-10x+41}}=
    \int{\frac{du}{u}}-7\int{\frac{dx}{(x-5)^{2}+16}}=
    \ln{|u|}-\frac{7}{4}\arctan{\left(\frac{x-5}{4}\right)}+C=$ \\
    
    $\ln{|x^{2}-10x+41|}-\frac{7}{4}\arctan{\left(\frac{x-5}{4}\right)}+C$ \\
    \pagebreak
  \item{$\int{x^{2}\sin{(3x)}}\,dx$} \\
  
    Let $u=x^{2}\therefore du=2xdx$ and let $v=-\frac{1}{3}\cos{(3x)}\therefore dv=\sin{(3x)}dx$ \\
    
    $\int{x^{2}\sin{(3x)}}\,dx=
    \int{u}\,dv=
    \frac{2}{3}\int{x\cos{(3x)}}\,dx-\frac{1}{3}x^{2}\cos{(3x)}$ \\
    
    Let $\alpha=x\therefore d\alpha=dx$ and let $\beta=\sin{3x}\therefore d\beta=3\cos{(3x)}dx$ \\
    
    $\frac{2}{3}\int{x\cos{(3x)}}\,dx-\frac{1}{3}x^{2}\cos{(3x)}=
    \frac{2}{3}\int{\alpha}\,d\beta-\frac{1}{3}x^{2}\cos{(3x)}=
    \frac{2}{3}\left(x\sin{(3x)}-\int{\sin{(3x)}}\,dx\right)-\frac{1}{3}x^{2}\cos{(3x)}=$ \\
    
    $\frac{2}{3}\left(x\sin{(3x)}+\frac{1}{3}\cos{(3x)}\right)-\frac{1}{3}x^{2}\cos{(3x)}+C$ \\
  
  \item{$\int{\arcsin{(3x)}}\,dx$} \\
  
    Let $u=\arcsin{(3x)}\therefore du=\frac{3}{\sqrt{1-9x^{2}}}dx$ and let $v=x\therefore du=dx$ \\
    
    $\int{\arcsin{(3x)}}\,dx=
    \int{u}\,dv=
    x\arcsin{(3x)}-\int{\frac{3x}{\sqrt{1-9x^{2}}}\,dx}$ \\
    
    Let $\alpha=1-9x^{2}\therefore d\alpha=-18xdx$ \\
    
    $x\arcsin{(3x)}-\int{\frac{3x}{\sqrt{1-9x^{2}}}\,dx}=
    x\arcsin{(3x)}+\frac{1}{6}\int{\frac{du}{\sqrt{u}}}=
    x\arcsin{(3x)}+\frac{1}{3}\sqrt{u}+C=$ \\
    
    $x\arcsin{(3x)}+\frac{1}{3}\sqrt{1-9x^{2}}+C$ \\
    
  \item{$\int{\sin{(3x)}\cos{(2x)}}\,dx$} \\ 
  
    $\int{\sin{(3x)}\cos{(2x)}}\,dx=
    \frac{1}{2}\left(\int{\sin{x}}\,dx+\int{\sin{(5x)}}\,dx\right))=
    -\frac{1}{2}\cos{x}-\frac{1}{10}\cos{(5x)}+C$ \\
    
  \item{$\int{\sec^{3}{(7x)}\tan{(7x)}}\,dx$} \\
  
    Let $\theta=7x\therefore d\theta=7dx$ \\
    
    $\int{\sec^{3}{(7x)}\tan{(7x)}}\,dx=
    \frac{1}{7}\int{\sec^{3}{\theta}\tan{\theta}}\,d\theta$ \\
    
    Let $u=\sec{\theta}\therefore du=\sec{\theta}\tan{\theta}d\theta$ \\
    
    $\frac{1}{7}\int{u^{2}}\,du=
    \frac{1}{21}u^{3}+C=
    \frac{1}{21}\sec^{3}{(7x)}+C$ \\
    
  \item{$\int{\sin^{2}{(5x)}\cos^{2}{(5x)}}\,dx$} \\
  
    $\int{\sin^{2}{(5x)}\cos^{2}{(5x)}}\,dx=
    \int{\left(\frac{1-\cos{(10x)}}{2}\right)\left(\frac{1+\cos{(10x)}}{2}\right)}\,dx=
    \frac{1}{4}\left(\int{}\,dx-\int{\cos^{2}{(10x)}}\,dx\right)= $ \\
    
    $\frac{1}{4}\left(\int{}\,dx-\frac{1}{2}\left(\int{}\,dx+\int{\cos{(20x)}}\,dx\right)\right)=
    \frac{1}{8}x-\frac{1}{160}\sin{(20x)}+C$ \\
    
  \item{$\int{e^{3x}\cos{x}}\,dx$} \\
    
    Let $u=e^{3x}\therefore du=3e^{3x}dx$ and let $v=\sin{x}\therefore dv=\cos{x}dx$ \\
    
    $\int{e^{3x}\cos{x}}\,dx=
    \int{u}\,dv=
    e^{3x}\sin{x}-3\int{e^{3x}\sin{x}}\,dx$ \\
    
    Let $\alpha=-\cos{x}\therefore d\alpha=\sin{x}dx$ \\
    
    $e^{3x}\sin{x}-3\int{e^{3x}\sin{x}}\,dx=
    e^{3x}\sin{x}-3\int{u}\,d\alpha=
    e^{3x}\sin{x}-3\left(3\int{e^{3x}\cos{x}}\,dx-e^{3x}\cos{x}\right)$ \\
    
    Let $\beta=\int{e^{3x}\cos{x}}\,dx$ \\
    
    $\beta=e^{3x}\sin{x}-9\beta+3e^{3x}\cos{x}\rightarrow 
    10\beta=e^{3x}\sin{x}+3e^{3x}\cos{x}\rightarrow
    \beta=\frac{1}{10}e^{3x}\sin{x}+\frac{3}{10}e^{3x}\cos{x}$
    
  \item{$\int_{0}^{\frac{\pi}{6}}{x\cos{(2x)}}\,dx$} \\
  
    Let $u=x\therefore du=dx$ and let $v=\frac{1}{2}\sin{(2x)}\therefore dv=\cos{(2x)}dx$ \\
    
    $\int_{0}^{\frac{\pi}{6}}{x\cos{(2x)}}\,dx=
    \int{u}\,dv=
    \frac{1}{2}[x\sin{(2x)}]_{0}^{\frac{\pi}{6}}-\frac{1}{2}\int_{0}^{\frac{\pi}{6}}{\sin{(2x)}}\,dx=
    \frac{1}{2}[x\sin{(2x)}]_{0}^{\frac{\pi}{6}}+\frac{1}{4}[\cos{(2x)}]_{0}^{\frac{\pi}{6}}=$ \\ 
    
    $\frac{\sqrt{3}\pi-3}{24}$ \\
    
  \item{$\int{\cos{(5x)}\cos{(4x)}}\,dx$} \\
  
    $\int{\cos{(5x)}\cos{(4x)}}\,dx=
    \frac{1}{2}\left(\int{\cos{(9x)}}\,dx+\int{\cos{(x)}}\,dx\right)=
    \frac{1}{18}\sin{(9x)}+\frac{1}{2}\sin{x}+C$ \\
    
    \pagebreak
  \item{(2004 Form B - AB 2) (Calc) \\
        For $0\leq t\leq 31$, the rate of change of the number of mosquitoes on Tropical Island at time $t$ days is modeled by $R(t)=5\sqrt{t}\cos{\left(\frac{t}{5}\right)}$ mosquitoes per day. There are 1000 mosquitoes on Tropical Island at time $t=0$.}
    \begin{enumerate}
      \item{Show that the number of mosquitoes is increasing at time $t=6$.} \\
      
        $R(6)=5\sqrt{6}\cos{\left(\frac{6}{5}\right)}=4.438$ \\
        
        The number of mosquitoes is increasing at time $t=6$ because $R(t)$ is positive at that value of $t$.
        
      \item{At time $t=6$, is the number of mosquitoes increasing at an increasing rate, or is the number of mosquitoes increasing at a decreasing rate? Give a reason for your answer.} \\ 
        
        $\frac{dR}{dt}=\frac{5\cos{\left(\frac{x}{5}\right)}}{2\sqrt{x}}-\sqrt{x}\sin \left(\frac{x}{5}\right)$ \\
        
        $R'(6)=\frac{5\cos{\left(\frac{6}{5}\right)}}{2\sqrt{6}}-\sqrt{6}\sin \left(\frac{6}{5}\right)=-1.913$ \\
        
        The number of mosquitoes are increasing at a decreasing rate because $R'(t)$ is negative at $t=6$. \\
        
      \item{According to the model, how many mosquitoes will be on the island at time $t=31$? Round your answer to the nearest whole number. } \\
      
        $\int_{0}^{31}{5\sqrt{t}\cos{\left(\frac{t}{5}\right)}\,dt=−35.665$ \\
        
        $1000−35.665\approx 964$ mosquitoes. \\
        
        964 mosquitoes will be on the island at $t=31$.
        
      \item{To the nearest whole number, what is the maximum number of mosquitoes for $0\leq t\leq 31$? Show the analysis that leads to your conclusion. } \\
      
        $R(t)=0$ at $t=0\,, t=2.5\pi\,, t=7.5\pi$ \\
        
        
        
        
    \end{enumerate} 
\end{enumerate}
\end{document}
