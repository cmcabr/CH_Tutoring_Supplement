% Document Metadata
\documentclass[10pt,letterpaper]{report}
\usepackage[utf8]{inputenc}
% Use for Arial Font \usepackage{helvet}
%  \renewcommand{\familydefault}{\sfdefault}
% Use for Times New Roman Font \usepackage{mathptmx}

\usepackage[none]{hyphenat}
\usepackage{tikz, textcomp, gensymb, graphicx, mathtools, amssymb, amsthm, hyperref}
  \hypersetup{
      colorlinks=true,
      linkcolor=blue,
      filecolor=magenta,
      urlcolor=blue,
      }
  \graphicspath{ {/home/lowebang/Pictures/} }
\usepackage[letterpaper]{geometry}
  \geometry{top=1in, bottom=1in, left=1in, right=1in}
\usepackage{fancyhdr}
  \pagestyle{fancy}
  \lhead{}
  \chead{}
  \rhead{Cabrera \thepage}
  \lfoot{}
  \cfoot{}
  \rfoot{\LaTeX}
  \renewcommand{\headrulewidth}{1pt}
  \renewcommand{\footrulewidth}{1pt}
  \setlength\headsep{0.333in}

% Command to Circle String
\newcommand*\circled[1]{\tikz[baseline=(char.base)]{
            \node[shape=circle,draw,inner sep=2pt] (char) {#1};}}

% Command to Set Oval Around String
\newcommand{\mymk}[1]{%
  \tikz[baseline=(char.base)]\node[anchor=south west, draw,rectangle, rounded corners, inner sep=2pt, minimum size=7mm,
  text height=2mm](char){\ensuremath{#1}} ;}

\title{Calculus BC -- Worksheet on Arc Length and Review }
\author{Craig Cabrera}
\date{2 March 2022}

\begin{document}
\maketitle
\begin{center}
  \textbf{\underline{Relevant Formulas and Notes:}}
\end{center}

Area Bounded by Two Graphs: \\
$$A=\int_{a}^{b}{f(x)-g(x)}\,dx$$ \\

$$A=\int_{c}^{d}{f(y)-g(y)}\,dy$$ \\

Arc Length: \\
$$s=\int _a^b\sqrt{1+\left(f\:'\left(x\right)\right)^{2}}\,dx$$ \\

$$s=\int _c^d\sqrt{1+\left(f\:'\left(y\right)\right)^{2}}\,dy$$ \\

\pagebreak 


\noindent Work the following on \textbf{\underline{notebook paper}}. \\
\noindent On problems 1 -- 3, find the arc length of the graph of the function over the indicated interval. Do \textbf{\underline{not}} use your calculator on problems 1 - 3. \\

\begin{enumerate}
  \item{$y=\frac{2}{3}x^{\frac{3}{2}}+1, \, [0, 1]$ \\}
  
    $\frac{dy}{dx}=\frac{2*3}{3*2}x^{\frac{1}{2}}=\sqrt{x}$ \\
    
    $s=\int_{0}^{1}{\sqrt{1+\left(\sqrt{x}\right)^{2}}}\,dx=\int_{0}^{1}{\sqrt{1+x}}\,dx$ \\
    
    Let $u=1+x\therefore du=dx$ \\
    
    $s=\int_{0}^{1}{\sqrt{1+x}}\,dx=\int_{1}^{2}{\sqrt{u}}\,du=\left[\frac{2}{3}u^{\frac{3}{2}}\right]_{1}^{2}=\frac{2}{3}\left(2\sqrt{2}-1\sqrt{1}\right)=\frac{4\sqrt{2}-2}{3}\approx\frac{5.656-2}{3}=\frac{3.656}{3}\approx 1.219$ \\
    
  \item{$y=\frac{3}{2}x^{\frac{2}{3}}, \, [1,8]$ \\}
  
    $\frac{dy}{dx}=\frac{3*2}{2*3}x^{-\frac{1}{3}}=\frac{1}{\sqrt[3]{x}}$ \\
    
  \item{$y=\ln{\left(\sin{x}\right)}, \, \left[\frac{\pi}{4}, \frac{3\pi}{4}\right]$ \\}
  
    $\frac{dy}{dx}=\frac{1}{\sin{x}}\frac{d}{dx}\left(\sin{x}\right)=\frac{\cos{x}}{\sin{x}}=\cot{x}$ \\
    
    $s=\int_{\frac{\pi}{4}}^{\frac{3\pi}{4}}{\sqrt{1+\cot^{2}{x}}}\,dx=\int_{\frac{\pi}{4}}^{\frac{3\pi}{4}}{\csc{x}}\,dx$ \\
    
    Let $u=\tan{\left(\frac{x}{2}\right)}$ \\
    
    $\csc{x}=\frac{1}{\sin{x}}=\frac{\cos^{2}{\left(\frac{x}{2}\right)}+\sin^{2}{\left(\frac{x}{2}\right)}}{\sin{x}}=\frac{1}{2\cos{\left(\frac{x}{2}\right)}\sin{\left(\frac{x}{2}\right)}}=\frac{1+u^{2}}{2u}$ \\
    
    $du=\frac{1+u^{2}}{2}dx\rightarrow dx=\frac{2}{1+u^{2}}du$ \\
\end{enumerate}
\end{document}
