\documentclass[12pt, letterpaper]{report}
\usepackage[letterpaper]{geometry}
  \geometry{top=1in, bottom=1in, left=1in, right=1in}
\usepackage[utf8]{inputenc}
\usepackage{pdfpages}
\usepackage[none]{hyphenat}
\usepackage{fancyhdr}
  \pagestyle{fancy}
  \lhead{}
  \chead{}
  \rhead{Watts et. al. \thepage}
  \lfoot{}
  \cfoot{}
  \rfoot{}
  \renewcommand{\headrulewidth}{1pt}
  \renewcommand{\footrulewidth}{1pt}
  \setlength\headsep{0.333in}
\usepackage{etoolbox, titlesec, setspace}
  \patchcmd{\chapter}{\thispagestyle{plain}}{\thispagestyle{fancy}}{}{}
  \titleformat{\chapter}[display]{\normalfont\huge\bfseries}{\chaptertitlename\ \thechapter}{20pt}{\Huge}
  \titlespacing*{\chapter}{0pt}{0pt}{40pt}
  \linespread{1.25}
\usepackage{textcomp, gensymb, mathtools , amssymb , amsthm, graphicx, hyperref, xcolor}
  \graphicspath{ {/home/lowebang/Pictures/} }
  \hypersetup{
      colorlinks,
      linkcolor={red!50!black},
      citecolor={blue!50!black},
      urlcolor={blue!80!black}
  }

\title{Conservation of Energy Laboratory Worksheet}
\author{Nick Watts, Jaden Ho, Craig Cabrera}
\date{19 January 2022}

\begin{document}
  \maketitle
  \tableofcontents
\pagebreak
  \chapter{General Instruction}
    \section{Purpose of Laboratory}
      \par The purposes of the following laboratory project is the following:
      \begin{itemize}
        \item{To identify the functions of one-dimensional kinematics in action in a simulated environment, including the relation between displacement and position, velocity, and acceleration in one axis.}
        \item{To connect the ideas of one-dimensional kinematics to the functions of kinetic and potential energy forms, including spring potential energy, rotational kinetic energy through motion of a wheel, and translational kinetic energy through the forces applied to a projectile.}
        \item{To extend the functions of kinetic energy to the forces of static and kinetic friction.}
        \item{To analyze the functions of conservation of energy in the aforementioned attributes of the system.}
      \end{itemize}
    \section{Preparation and Background of Laboratory}
      \par The topics discussed within this section relate to the ideas of one-dimensional kinematics, systems, energy, and work, so the following should be reflected in the following hypotheses and questions, and each student should have covered this. Each of these topics should tie together under the concept of Conservation of Energy, also discissed previously by the students covering this lab.
    \pagebreak
    \section{Required Materials for Experimentation}
      \noindent The following lab requires these materials:
      \begin{itemize}
        \item{Provided/Required Materials:}
        \begin{itemize}
          \item{Graphing Space and Table for Data (Sections 2.3 -- 2.4)}
          \item{Smooth, Flat Surface, 1 Meter (Preferrably Cardboard)}
          \item{Rough, Flat Surface, 1 Meter (Preferrably Sandpaper or Gritty Stone)}
          \item{Any Item to Angle Incline}
          \item{Stopwatch}
          \item{Toy Car}
          \item{Weight}
        \end{itemize}
        \item{Optional/Non-Provided Materials}
          \begin{itemize}
            \item{Formula Sheet}
            \item{Calculator}
            \item{Markers}
          \end{itemize}
      \end{itemize}
    \section{Safety Procedures}
      \begin{itemize}
        \item{Follow proper lab safety protocol.}
      \end{itemize}
\pagebreak
  \chapter{Laboratory Instruction}
    \section{Proceedings}
      \par Using the given materials, set up a ramp system that tests the effects of friction on an object when released from rest. You will use both the rough and smooth ramp to document the time it takes for the object to reach the bottom of the ramp, and the distance it travels after leaving the end of the ramp. You will also document the distance from the ramp the object was pulled back.
      \section{Formation of Hypothesis Based Upon Demonstration}
      \par Before we begins our lab demonstration, let us first produce a hypothesis. Write what you think will happen when the car is sent down the ramp, including your assumptions of the values of velocity, acceleration, and angle of the ramp. Then, write what you think the relation between these values and the values of energy are. \\

      \noindent \rule{6.5in}{0.15mm} \\

      \noindent \rule{6.5in}{0.15mm} \\

      \noindent \rule{6.5in}{0.15mm} \\

      \noindent \rule{6.5in}{0.15mm} \\

      \noindent \rule{6.5in}{0.15mm} \\

      \noindent \rule{6.5in}{0.15mm} \\

    \pagebreak
    \section{Data Collection and Experimentation}
      \par Given the materials provided to you, set the following items as described:
      \begin{enumerate}
        \item{Set the item for angle underneath either of the surfaces, and set the toy car at the top of the ramp. \textbf{Make sure in both setups that the surfaces are both at the same angle, and measure the height of the ramp.}}
        \item{Use your markers to set marks at one meter distances from the top of the ramp, from one to four meters.}
        \item{Record how far back the car is pulled back from the ramp, then release the car from the top of the ramp.}
        \item{Record the time in which the car reached each meter marker in your table.}
        \item{Repeat steps 2-4 for three different iterations of the same trial.}
        \item{For the surface that you have not tested, repeat steps 1-5 for this material.}
        \item{Once done, continue to Section 2.4.} \\
      \end{enumerate}
      Height, Pull-Back, and Angle: \_\_\_\_\_\_$h=$\_\_\_\_\_\_\_\_\_\_\_\_\_\_\_\_\_\_\_\_\_\_\_\_$\Delta x=$\_\_\_\_\_\_\_\_\_\_\_\_\_\_\_\_\_\_\_\_$\theta$=\_\_\_\_\_\_\_\_\_\_\_\_ \\
      Position Table for Smooth Surface:
      \begin{center}
        \begin{tabular}{| c | c | c | c | c |}
          \hline
          \textbf{Time (sec.)} & \textbf{Trial 1} & \textbf{Trial 2} & \textbf{Trial 3} & \textbf{Average} \\
          \hline
          \textbf{1 m} & & & & \\
          \hline
          \textbf{2 m} & & & & \\
          \hline
          \textbf{3 m} & & & & \\
          \hline
          \textbf{4 m} & & & & \\
          \hline
          \textbf{5 m} & & & & \\
          \hline
          \textbf{6 m} & & & & \\
          \hline
        \end{tabular}
      \end{center}
      Position Table for Rough Surface:
      \begin{center}
        \begin{tabular}{| c | c | c | c | c |}
          \hline
          \textbf{Time (sec.)} & \textbf{Trial 1} & \textbf{Trial 2} & \textbf{Trial 3} & \textbf{Average} \\
          \hline
          \textbf{1 m} & & & & \\
          \hline
          \textbf{2 m} & & & & \\
          \hline
          \textbf{3 m} & & & & \\
          \hline
          \textbf{4 m} & & & & \\
          \hline
          \textbf{5 m} & & & & \\
          \hline
          \textbf{6 m} & & & & \\
          \hline
        \end{tabular}
      \end{center}
    \pagebreak
    \section{Analysis of Recorded Data}
      The following calculations can be written on the back of this page, and will be done with your trial averages. Each of these questions should also be calculated for \textbf{both} trials unless expressed not to do so.
      \begin{enumerate}
        \item{Given the height of the top of the ramp and the mass of the car, what was the initial gravitational potential energy?}
        \item{A variable $k$ is given for the constant of the spring in the toy car. Using this variable and your distance for car pull-back, derive an expression for the elastic potential energy of the toy car.}
        \item{Based on your data from the lab, estimate the instantaneous velocity at the given intervals and times:}
        \begin{center}
          \begin{tabular}{| c | c | c | c | c | c | c |}
            \hline
            \textbf{Position (m)} & 1 m & 2 m & 3 m & 4 m & 5 m & 6 m \\
            \hline
            \textbf{Velocity (m/sec.)} & & & & & & \\
            \hline
          \end{tabular}
        \end{center}
        \item{Use each calculation for velocity to determine instantaneous acceleration at the given intervals and times:}
        \begin{center}
          \begin{tabular}{| c | c | c | c | c | c | c |}
            \hline
            \textbf{Position (m)} & 1 m & 2 m & 3 m & 4 m & 5 m & 6 m \\
            \hline
            \textbf{Velocity (m/sec$^{2}$.)} & & & & & & \\
            \hline
          \end{tabular}
        \end{center}
        \item{Given the velocity at the bottom of the ramp ($x$ = 1 m) and the mass of the cart, derive an expression for the maximum translational kinetic energy.}
        \item{Define the formula for rotational kinetic energy.}
        \item{Using your expressions for energies, write an expression showing net kinetic energies.}
        \item{Using the value of net acceleration and the mass of the cart, determine the coefficient of kinetic friction. (Note: $\mu$ should be set as the coefficient of a frictionally negligent surface to determine its value.)}
      \end{enumerate}
    \pagebreak
    .
    \pagebreak
    \section{Interpolation of Recorded Data}
      \par Using the information you have gathered and recorded, determine the line of each variable for the car below. (Optional - Also label each line with an equation of best fit). Make sure to include scale, labels, and a title. Use the following color code for each term:
      \begin{itemize}
        \item{Displacement - Red}
        \item{Velocity - Green}
        \item{Acceleration - Blue}
        \item{Net Kinetic Energy (Estimate) - Yellow}
        \item{Net Potential Energy (Estimate) - Orange}
      \end{itemize}
      (The following blank graphs are optional; it can be graphed by the student electronically or on paper.) \\
      \pagebreak
      \begin{center}
        Graph of Smooth Surface: \\
        \includegraphics[scale=0.5]{graph2.png}
      \end{center}
      \vfill
      \begin{center}
        Graph of Rough Surface: \\
        \includegraphics[scale=0.5]{graph2.png}
      \end{center}
      \pagebreak
  \chapter{Deduction}
    \section{Conclusions}
      \begin{enumerate}
        \item{Describe the car's motion in terms of the position.}
        \item{How does the motion of the car relate to the car's translational kinetic energy?}
        \item{Describe the car's acceleration in terms of motion.}
        \item{Was there any acceleration in the car's motion?}
        \item{What was the range of acceleration for the toy car?}
        \item{How does the acceleration of the toy car relate to its rotational kinetic energy?}
        \item{Assume $k=300$ for the spring constant of the car. Solve for rotational kinetic energy, then determine the maximum kinetic and potential energies, given the values you currently have and by measuring the radius of the wheel.}
        \item{How do you believe the surface of the ramp impacted the static friction of the object?}
        \item{How do you believe the rotational kinetic energy impacted the kinetic friction of the object?}
      \end{enumerate}
\pagebreak
    \section{Leading Questions}
      \begin{enumerate}
        \item{Describe how the lab relates to the discussions/assignments you have been given in class.}
        \item{Describe the motion of the car in terms of its energy.}
        \item{Why do you think each graph line follows the path it does?}
        \item{According to the idea of conservation of energy, why would the potential and kinetic energy graphs be opposite?}
        \item{Explain why the topic of kinematics is of such importance for the calculations of energies.}
        \item{How were you limited in this experiment? Where were any errors in your values?}
        \item{Comments:}
      \end{enumerate}
\pagebreak
\begin{center}
  Answer Key
    \end{center}
    2.2 -- Answers may vary, and may change in topic depending on the background knowledge of the student. \\
    2.3 -- Answers may vary. Values should come from the student's own recordings. \\
    2.4:
    \begin{enumerate}
      \item{Formula for Gravitational Potential Energy -- $mgh$}
      \item{Formula for Elastic Potential Energy -- $\frac{1}{2}k\Delta x^{2}$}
      \item{Instantaneous Velocity Formula -- $\frac{dx}{dt}$}. Values should have similar difference from each other.
      \item{Acceleration Formula -- $\frac{dv}{dt}$. Difference in velocity should be similar to the constant acceleration values found.}
      \item{Formula for Translational Kinetic Energy -- $\frac{1}{2}mv_{1}^{2}$}
      \item{Formula for Rotational Kinetic Energy -- $\frac{1}{2}I\omega^{2}$}
      \item{$2mgh+k\Delta x^{2}=mv_{1}^{2}+I\omega^{2}$}
      \item{$m_{2}a_{net_{2}}=\mu m_{1}a_{net_{1}}$, where 1 is representative of the smooth surface trial and 2 is representative of the rough surface trial.}
    \end{enumerate}
    \pagebreak
    3.1:
    \begin{enumerate}
      \item{Focus on the change in velocity and its impact on rate of change of position.}
      \item{Either refer to the formula or use the graph to show a change in slope of one based on the other.}
      \item{Focus on acceleration and its impact on the rate of change of velocity.}
      \item{Most likely the acceleration will be uniform.}
      \item{Give margin of error.}
      \item{Describe acceleration's role in both kinetic friction on both surfaces and angular velocity.}
      \item{$9.8mh + 300\Delta x^{2}=mv_{1}^{2}+I\omega^{2}$; values should be more than enough to determine maximums; maximums for both should have minimal difference.}
      \item{Describe how rough surfaces impact the amount of friction on a wheel.}
      \item{Describe the force of inertia and the change it causes on kinetic energy, or describe the friction of the wheels on each surface.}
    \end{enumerate}
    3.2:
    \begin{enumerate}
      \item{Describe similar questions to the ones taught in class relevant to the subject.}
      \item{Describe the change in position relative to change in kinetic energy.}
      \item{Describe possible relationships between graphs, i.e. velocity vs. energy, mass vs. friction and potential energy}
      \item{Potential and kinetic energy are inverses of one another which have inverse relations.}
      \item{Velocity vs. kinetic energy relation, displacement vs. potential energy relation}
      \item{Describe any lab mishaps.}
    \end{enumerate}
\end{document}
