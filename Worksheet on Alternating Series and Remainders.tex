% Document Metadata
\documentclass[10pt,letterpaper]{report}
\usepackage[utf8]{inputenc}
% Use for Arial Font \usepackage{helvet}
%  \renewcommand{\familydefault}{\sfdefault}
% Use for Times New Roman Font \usepackage{mathptmx}

\usepackage[none]{hyphenat}
\usepackage{tikz, textcomp, gensymb, graphicx, mathtools, amssymb, amsthm, hyperref, multicol}
  \hypersetup{
      colorlinks=true,
      linkcolor=blue,
      filecolor=magenta,
      urlcolor=blue,
      }
  \graphicspath{ {/home/lowebang/Pictures/} }
\usepackage[letterpaper]{geometry}
  \geometry{top=1in, bottom=1in, left=1in, right=1in}
\usepackage{fancyhdr}
  \pagestyle{fancy}
  \lhead{}
  \chead{}
  \rhead{Cabrera \thepage}
  \lfoot{}
  \cfoot{}
  \rfoot{\LaTeX}
  \renewcommand{\headrulewidth}{1pt}
  \renewcommand{\footrulewidth}{1pt}
  \setlength\headsep{0.333in}

% Command to Circle String
\newcommand*\circled[1]{\tikz[baseline=(char.base)]{
            \node[shape=circle,draw,inner sep=2pt] (char) {#1};}}

% Command to Set Oval Around String
\newcommand{\mymk}[1]{%
  \tikz[baseline=(char.base)]\node[anchor=south west, draw,rectangle, rounded corners, inner sep=2pt, minimum size=7mm,
  text height=2mm](char){\ensuremath{#1}} ;}

\title{Calculus BC -- Worksheet on Alternating Series and Remainders}
\author{Craig Cabrera}
\date{10 March 2022}

\begin{document}
\maketitle
\begin{center}
  \textbf{\underline{Relevant Formulas and Notes:}}
\end{center}

\pagebreak 


Work the following on \textbf{\underline{notebook paper}}.
\begin{enumerate}
  \item{Approximate the sum, $S$, of the series $\sum_{n=0}^{\infty}{\frac{(-1)^{n}}{n!}}$ by using its first five terms, and explain why your estimate differs from the actual value by less than .009. Then use your results to find an interval in which $S$ must lie. \\}
  
    $S=\sum_{n=0}^{\infty}{\frac{(-1)^{n}}{n!}}\approx \sum_{n=0}^{4}{\frac{(-1)^{n}}{n!}}= 1 - 1 + \frac{1}{2} - \frac{1}{6} + \frac{1}{24} = \frac{3}{8}$ \\
    
    $|\text{error}| = |S-S_{4}| < |a_{5}|$ \\
    
    $|\text{error}| < |\frac{(-1)^5}{5!}| = |-\frac{1}{120}| \approx 0.00833 < 0.009 $ \\
    
    $\therefore |\text{error}| < 0.009 $ by the Alternatng Series Remainder. \\
    
    $S_{4} - \frac{1}{120} = 0.366 < S < S_{4} + \frac{1}{120} = 0.383 \therefore S $ lies in the interval $(0.366, 0.383)$ or $(\frac{11}{30}, \frac{23}{60})$ \\
    
  \item{}
    
  \item{Approximate the sum of the convegent series $\sum_{n=0}^{\infty}{\frac{(-1)^{n}}{2^{n}n!}}$ so that the error will be less than $\frac{1}{1000}?$ How many terms were needed? What are the properties of the terms of this series that guarantee that your approximation is within $\frac{1}{1000}$ of the exact value? Justify your answer. \\}
  
    $S = \sum_{n=0}^{\infty}{\frac{(-1)^{n}}{2^{n}n!}} \approx \sum_{n=0}^{4}{\frac{(-1)^{n}}{2^{n}n!}} = 1 - \frac{1}{2} + \frac{1}{8} - \frac{1}{48} + \frac{1}{384} \approx \frac{233}{384}$ \\
    
    $|\text{error}| = |S - S_{4}| < |\frac{(-1)^{5}}{32*120}| = |-\frac{1}{3840}| < \frac{1}{1000}$ \\
    
    $\therefore |\text{error}| < 1000$  by the Alternating Series Remainder. \\
    
  \item{$ $}
    
  \item{Approximate the sum of the convegent series $\sum_{n=0}^{\infty}{\frac{(-1)^{n+1}}{n^{3}}$ so that the error will be less than $\frac{1}{1200}?$ How many terms were needed? What are the properties of the terms of this series that guarantee that your approximation is within $\frac{1}{1200}$ of the exact value? Justify your answer. \\}
    
  
\end{enumerate}
\end{document}
